\documentclass{article}
\begin{document}
   \section{Dimostrazione per Induzione}
   Dato $n$ qualunque se e' vera $P(n)$,allora e' vera anche $P(n+1)$
   \subsection{Base dell'induzione}
   Verificare $P(1)$
   \subsection{Passo Induttivo}
   Supponendo che $P(n)$ sia vero, ne consegue che $P(n+1)$ e' vera.In altri termini
   $P(n)\to P(n+1)$ vera.
   \section{Esempio dimostrazione per induzione}
   \begin{equation}
    \sum_{i=1}^n=\frac{n(n+1)}{2}
   \end{equation}
   \subsection{Base dell induzione}
   OK
   \subsection{Passo Induttivo}
   Prendiamo $k\in N$ qualunque
   \begin{equation}
     P(k)=\sum_{i=1}^k i= \frac{k(k+1)}{2}\
   \end{equation}
   \begin{equation}
     P(k+1)=\sum_{i=1}^{k+1} i= \frac{(k+1)(k+2)}{2}
   \end{equation}
   \begin{equation}
    \sum_{i=1}^{k+1} i=\sum_{i=1}^k i+(k+1)
   \end{equation}
   Equivale a dire
   \begin{equation}
    \frac{(k+1)(k+2)}{2}=\frac{(k+1)(k+2)}{2}
   \end{equation}
\end{document}
