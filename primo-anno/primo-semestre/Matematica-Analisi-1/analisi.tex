\documentclass{article}
\usepackage{amsmath}
\usepackage{amssymb}
\usepackage{amsfonts}
\title{Matematica Analisi}
\begin{document}
   \section{Principio di Induzione con Esempio}
   \begin{flushleft}
    Il principio di Induzione serve a dimostrare che data una espressione $P(n)$ e supponendo che sia vera allora e' vero anche $P(n+1)$ e cosi via
   \end{flushleft}
   \subsection{Esempio dimostrazione con il principio di induzione}
   \begin{flushleft}
    Dimostriamo per induzione:
    \begin{equation}
        (a+b)^n= \sum_{k=0}^{n}
        \begin{pmatrix}
            n\\
            k
        \end{pmatrix}a^kb^{n-k}
    \end{equation}
    Ricordiamo qualche formula:
    \begin{equation}
        \begin{pmatrix}
            n\\
            k
        \end{pmatrix}=\frac{n!}{k!(n-k)!}
    \end{equation}
    \begin{equation}
        \begin{pmatrix}
            n+1\\
            k
        \end{pmatrix}=\begin{pmatrix}
                        n\\
                        k-1
                      \end{pmatrix}+\begin{pmatrix}
                                       n\\
                                       k
                                     \end{pmatrix}
    \end{equation}
    Ora andiamo a risolvere l esercizio. Il principio di induzione applica 2 "passi":
    \begin{itemize}
    \item il passo base che prendendo un numero a caso verifichiamo che $P(1)$ (1 in questo caso) sia verificata.
    \item il passo induttivo che supponendo che $P(n)$ sia vera, confermiamo anche che $P(n+1)$ sia vera per qualsiasi n (non qualsiasi! Dipende dall insieme in cui ci troviamo e cosa richiede l esercizio)
    \end{itemize}
   \subsection{Passo Base}
   Applichiamo il passo base mettendo come valore 1 $n=1$ oppure $P(1)$
   \begin{equation}
        (a+b)^1= \sum_{k=0}^{1}
        \begin{pmatrix}
            n\\
            k
        \end{pmatrix}a^kb^{n-k}
   \end{equation}
   Completando i semplici calcoli notiamo che entrambe le espressioni sia quella di destra che di sinsitra riportano 1 come valore finale
   \subsection{Passo Induttivo}
    Ora dimostrare che $P(n) \implies P(n+1)$\\
    Riscriviamo la funzione $P(n+1)$
    \begin{equation}
        (a+b)^{n+1}= \sum_{k=0}^{n+1}
        \begin{pmatrix}
            n\\
            k
        \end{pmatrix}a^kb^{n-k}
    \end{equation}
    L'espressione sopra puo essere riscritta come:

    \begin{equation}
       (a+b)^{n+1}=(a+b)^n(a+b)
    \end{equation}

    \begin{equation}
        \sum_{k=0}^{n+1}
        \begin{pmatrix}
            n\\
            k
        \end{pmatrix}a^kb^{n-k}=(a+b)
        \sum_{k=0}^{n}
        \begin{pmatrix}
            n\\
            k
        \end{pmatrix}a^kb^{n-k}=
        \sum_{k=0}^{n}
        \begin{pmatrix}
            n\\
            k
        \end{pmatrix}a^{k+1}b^{n-k}+
        \sum_{k=0}^{n}
        \begin{pmatrix}
            n\\
            k
        \end{pmatrix}a^{k}b^{n+1-k}=
    \end{equation}
    NON HO CAPITO
    \begin{equation}
       = \sum_{k=1}^{n+1}
        \begin{pmatrix}
            n\\
            k-1
        \end{pmatrix}a^{k}b^{n+1-k}+
        \sum_{k=0}^{n}
        \begin{pmatrix}
            n\\
            k
        \end{pmatrix}a^{k}b^{n+1-k}+a^{n+1}+b^{n+1}=
    \end{equation}
    \begin{equation}
       = \sum_{k=1}^{n+1}(
        \begin{pmatrix}
            n\\
            k-1
        \end{pmatrix}+
        \begin{pmatrix}
            n\\
            k
        \end{pmatrix})
        a^{k}b^{n+1-k}\quad \square
    \end{equation}
    la formula si conclude applicando la formula $(3)$ alle matrici
   \end{flushleft}
   \section{Limiti}
   \subsection{Definizioni}
   Sia $A \subseteq \mathbb{R}$ con $A \neq 0$
   \begin{flushleft}
    $M_x \in \mathbb{R}$ si dice \textbf{MAGGIORANTE} di A se
    \begin{equation}
       \forall a \in A \quad a\leq M_x
    \end{equation}
    $m_x \in \mathbb{R}$ si dice \textbf{MINORANTE} di A se
    \begin{equation}
       \forall a \in A \quad a\geq m_x
    \end{equation}
    $M \in \mathbb{R}$ si dice \textbf{MASSIMO} di A se
    \begin{equation}
       \forall a \in A \quad a\leq M \quad M\in A
    \end{equation}
    $m \in \mathbb{R}$ si dice \textbf{MINIMO} di A se
    \begin{equation}
       \forall a \in A \quad a\geq m \quad m\in A
    \end{equation}
    A si dice \textbf{LIMITATO SUPERIORMENTE} se (A ha almeno un maggiorante)
    \begin{equation}
     \exists M_x \in \mathbb{R}\quad : \quad \forall a \in A \quad a\leq M_x
    \end{equation}
    A si dice \textbf{LIMITATO INFERIORMENTE} se (A ha almeno un minorante)
    \begin{equation}
     \exists m_x \in \mathbb{R}\quad : \quad \forall a \in A \quad a\geq m_x
    \end{equation}
   \end{flushleft}
    \begin{flushleft}
    \subsection{Esempi}
        Inizializziamo 2 insiemi:
        \begin{itemize}
           \item $A = \mathbb{N}=\{ 1,2,3,4...\}$
               \begin{itemize}
                 \item L'insieme dei maggioranti e' vuoto
                 \item L'insieme dei minoranti e' $(-\infty,o]$
                 \item A non e' limitato. min(A)=0, A non ha massimo
               \end{itemize}
           \item $A = (-2,1) \cup [2,3]$
               \begin{itemize}
                 \item L'insieme dei maggioranti e' $[3, +\infty)$
                 \item L'insieme dei minoranti e' $(-\infty,-2]$
                 \item A e' limitato. min(A)=3, A non ha minimo
               \end{itemize}
        \end{itemize}
    \end{flushleft}
    \subsection{Assioma di completezza}
    \begin{flushleft}
    Sia $A \subseteq \mathbb{R}$ con $A \neq 0$
        \begin{itemize}
            \item Se A e' limitato superiormente allora l' insieme dei maggioranti di A ha un elemento minimo
            che si dice \textbf{ESTREMO SUPERIORE} di A e si scrive $sup(A)$.
            \item Se A e' limitato inferiormente allora l' insieme dei minoranti di A ha un elemento massimo
            che si dice \textbf{ESTREMO INFERIORE} di A e si scrive $inf(A)$.
        \end{itemize}
        La definzione di sup e inf si estende algi insiemi non limitati:
        \begin{itemize}
            \item $sup(A)=+\infty$ se A non e' limitato superiormente
            \item $sup(A)=-\infty$ se A non e' limitato inferiormente
        \end{itemize}
        Osservazione
        \begin{itemize}
            \item Se A ha un massimo allora $max(A)=sup(A)$
            \item Se A ha un minimo allora $min(A)=inf(A)$
        \end{itemize}
        \subsection{Esempi}
        \begin{itemize}
            \item $A = \mathbb{N} \rightarrow inf(A)=min(A)=0$ e $sup(A)=+\infty$
            \item $A = (-2,1)\cup[2,3]\rightarrow inf(A)=-2$ e $sup(A)=max(A)=3$
            \item $A = \{\frac{1}{n}:n\in \mathbb{N}^+\}=\{1,\frac{1}{2},\frac{1}{3},\frac{1}{4},..\}\quad sup(A)=max(A)=1$,$inf(A)=0$\\ A non ha minimo.
        \end{itemize}
        \subsection{Proprieta e caratteristiche di sup e inf}
        \subsubsection{Proprieta dell inf}
        \begin{flushleft}
        Caratteristiche dell inf:
            \begin{itemize}
                \item $inf(A)=-\infty \iff \forall M \in \mathbb{R} \quad \exists a \in A \quad : \quad a<M$
                \begin{flushright}
                   Dato un qualsiasi numero M esiste un elemento di A piu' piccolo di M
                \end{flushright}
                \item $inf(A)=l\in \mathbb{R} \iff$
                \begin{equation}
                    \begin{cases}
                        \forall a \in A \quad l \leq a \quad \text{(l e' un minorante)}\\
                        \forall \epsilon >0 \quad \exists a \in A \quad : \quad a<l+\epsilon \quad \text{(ogni numero piu' grande di l non e' un minorante di A)}
                    \end{cases}
                \end{equation}
            \end{itemize}
        \end{flushleft}
        \subsubsection{Proprieta del sup}
        \begin{flushleft}
        Caratteristiche del sup:
            \begin{itemize}
                \item $inf(A)=+\infty \iff \forall M \in \mathbb{R} \quad \exists a \in A \quad : \quad a>M$
                \begin{flushright}
                   Dato un qualsiasi numero M esiste un elemento di A piu' grande di M
                \end{flushright}
                \item $inf(A)=l\in \mathbb{R} \iff$
                \begin{equation}
                    \begin{cases}
                        \forall a \in A \quad l \geq a \quad \text{(l e' un maggiorante)}\\
                        \forall \epsilon >0 \quad \exists a \in A \quad : \quad a>l-\epsilon \quad \text{(ogni numero piu' grande di l non e' un minorante di A)}
                    \end{cases}
                \end{equation}
            \end{itemize}
        \end{flushleft}
        \subsubsection{Esempi}
        % scrivere l esempio di x+5/x
        \subsection{Successione reale}
        \begin{flushleft}
            Una successione reale e' data una funzione dove questa parte da $n>n_0$ e continua per infinito
            \begin{equation}
                \{ a_n \}_{n \geq n_0}
            \end{equation}
            E.g
            \begin{equation}
                \{ \frac{(-1)^n}{n} \}_{n \geq 1}
            \end{equation}
        \end{flushleft}
        \subsection{Limiti di successione}
        \begin{flushleft}
            \begin{equation}
                \lim_{n \to +\infty} a_n
            \end{equation}
            \begin{flushleft}
               Vengono definiti in:
               \begin{itemize}
                   \item $ +\infty \quad \forall M \in \mathbb{R} \quad \exists N \geq n_0:\forall n \geq N \quad a_n > M$
                   \item $ -\infty \quad \forall M \in \mathbb{R} \quad \exists N \geq n_0:\forall n \geq N \quad a_n < M$
                   \item $ l=\mathbb{R} \quad \forall M \in \mathbb{R} \quad \exists N \geq n_0:\forall n \geq N \quad |l-a_n|<\epsilon$
                   \item non esiste altrimenti
               \end{itemize}
            \subsubsection{Esempi}
            \begin{flushleft}
               % esempi sui limiti e le proprieta sopra
            \end{flushleft}
            \subsection{Proprieta' dei limiti di successione}
                \subsubsection{Unicita' del limite}
                \begin{flushleft}
                    Tutti i casi mostriati nel paragrafo 2.7 sono distinti tra loro cioe
                    non e' possibile che piu di quei casi si avverano per lo stesso limite
                    % ci sarebbe da mettere la dimostrazione
                \end{flushleft}
                \subsubsection{bo}
                \begin{flushleft}
                    Se $\lim_{n \to \infty} a_n \quad a_n \in \mathbb{R}$ allora la sua successione e' limitata\\
                    $\{(-1)^n\}$ e' limitata ma non ha limiti
                \end{flushleft}
                \subsubsection{bo}
                \begin{flushleft}
                    se $\lim_{n \to \infty} a_n $ esiste allora ogni sottosuccessione $\{a_n\}$ ha lo stesso limite
                \end{flushleft}
            \end{flushleft}
                \subsubsection{Permanenza del segno}
                \begin{flushleft}
                    se $\lim_{n \to \infty} a_n  \in \mathbb{R} \cup \{+\infty\}$ allora definitivamente $a_n > 0$
                \end{flushleft}
                \subsubsection{Confronto}
                \begin{flushleft}
                    Dati i seguenti limiti
                    \begin{equation}
                        \lim_{n\to+\infty} a_n = l_1,\quad  \lim_{n\to+\infty} b_n = l_2
                    \end{equation}
                    Definitivamente $a_n \leq b_n \iff l_1 \leq l_2$
                \end{flushleft}
                \subsubsection{Doppio Confronto}
                \begin{flushleft}
                    Dati i seguenti limiti
                    \begin{equation}
                        \lim_{n\to+\infty} a_n = l,\quad  \lim_{n\to+\infty} b_n = l
                    \end{equation}
                    Definitivamente $a_n \leq c_n \leq b_n$ allora
                    \begin{equation}
                        \lim_{n\to+\infty} c_n=l
                    \end{equation}
                \end{flushleft}
                \subsubsection{bo}
                \begin{flushleft}
                    se $\lim_{n \to \infty} a_n =0 $ e $\{ b_n \}$ e' una sottosuccessione limitata allora
                    \begin{equation}
                        \lim_{n\to+\infty} a_n * b_n =0
                    \end{equation}
                \end{flushleft}
                \subsubsection{Algebra dei limiti}
                \begin{flushleft}
                    Sia $\lim_{n\to+\infty} a_n = l_1$, $\lim_{n\to+\infty} a_n = l_1$ $l_1,l_2 \in \mathbb{R}$ allora
                    \begin{equation}
                        \lim_{n\to+\infty} (a_n + b_n)=l_1+l_2
                    \end{equation}
                    \begin{equation}
                        \lim_{n\to+\infty} (a_n * b_n)=l_1*l_2
                    \end{equation}
                    \begin{equation}
                        \lim_{n\to+\infty} (\frac{a_n}{b_n})=(\frac{l_1}{l_2}) \quad l_2 \neq 0
                    \end{equation}
                \end{flushleft}
                \subsubsection{Limite di una successione monotona}
                \begin{flushleft}
                    se $a_n \leq a_{n+1}, \forall n \geq n_0$ (funzione crescente)
                    \begin{equation}
                        \lim_{n\to+\infty} a_n=sup\{a_n:n\geq n_0\}
                    \end{equation}
                    se $a_n \geq a_{n+1}, \forall n \geq n_0$ (funzione decrescente)
                    \begin{equation}
                        \lim_{n\to+\infty} a_n=sup\{a_n:n\geq n_0\}
                    \end{equation}
                \end{flushleft}
                \subsubsection{bo}
                \begin{flushleft}
                   Per le funzioni elementari
                    \begin{equation}
                        |x|,n^n,a^x,log(x),sen(x),cos(x),tg(x),arccos(x),arcsen(x),arctg(x)
                    \end{equation}
                    \\se $a \to l\in D$ allora $f(a_n)\to f(l)$
                \end{flushleft}
                \subsubsection{Esempi}
                \begin{equation}
                    \lim_{n\to+\infty} log(sin(\frac{1}{2}) + 2cos(\frac{2}{n^2+1})) = log(2)
                \end{equation}
                Sostituendo infinito a n riusciamo forse a trovare il risultato
                \subsection{Altre proprieta dei limiti}
                \begin{flushleft}
                    \begin{itemize}
                        \item $\lim_{n\to+\infty} a^n =$
                            \begin{itemize}
                             \item +$\infty$ se $a>1$
                             \item $1$ se $a=1$
                             \item $0$ se $a<1$
                            \end{itemize}
                        \item $\lim_{n\to+\infty} log_a(n)=$
                            \begin{itemize}
                             \item +$\infty$ se $a>1$
                             \item -$\infty$ se $0<a<1$
                            \end{itemize}
                        \item $\lim_{n\to+\infty} log_a(\frac{1}{n})=$
                            \begin{itemize}
                             \item -$\infty$ se $a>1$
                             \item +$\infty$ se $0<a<1$
                            \end{itemize}
                        \item Alcuni calcoli con infinito:
                            \begin{itemize}
                             \item $\frac{1}{+-\infty}=0$
                             \item $\frac{1}{0^+}=+\infty$
                             \item $\frac{1}{0^-}=-\infty$
                            \end{itemize}
                    \end{itemize}
                \end{flushleft}
                \subsection{Forme Indeterminate}
                \begin{flushleft}
                Sono forme che ancora non possiamo affrontare quindi per raggirarle dovremmo applicare le proprieta in base
                in base al limite
                 \begin{itemize}
                  \item $+-\infty*0$
                  \item $+\infty-\infty$
                  \item $\frac{+-\infty}{+-\infty}$
                  \item $1^{+\infty}$
                  \item $(0^+)^0$
                 \end{itemize}
                \end{flushleft}
        \end{flushleft}
        \subsection{Teorema (Criterio del rapporto)}
        \begin{flushleft}
            ${a_n}_n$ successione positiva tale che:
            \begin{equation}
                \lim_{x\to+\infty} \frac{a_n+1}{a_n}=L
            \end{equation}
            \begin{itemize}
                \item se $L>1$ allora $\lim_{n\to+\infty} a_n = +\infty$
                \item se $L<1$ allora $\lim_{n\to+\infty} a_n = 0$
                \item se $L=0$ allora il criterio e' inconcludente
            \end{itemize}
        \end{flushleft}
        \end{flushleft}
        \subsubsection{Esempio Teorema (criterio del rapporto)}
        \begin{flushleft}
            INSERIRE ESEMPIO
        \end{flushleft}
        \subsection{Confronti di infiniti}
        \begin{flushleft}
            \begin{equation}
                \lim_{n\to+\infty} a_n=\lim_{n\to+\infty} b_n=+\infty
            \end{equation}
            $\lim \frac{a_n}{b_n} =$
            \begin{itemize}
            \item $+\infty \to$ $a_n$ e' un infinito di ordine superiore a $b_n$
            \item $0 \to$ $a_n$ e' un infinito di ordine superiore a $a_n$
            \item $l\in \mathbb{R} \to$ $a_n$ e' un infinito di ordine superiore a $b_n$
            \end{itemize}
        \end{flushleft}
        \subsubsection{Infiniti in ordine}
        \begin{flushleft}
           Gli infiniti in ordine di "grandezza"
           \begin{equation}
               log_n(n), \quad n^b, \quad a^n, \quad n!, \quad n^n
           \end{equation}
        \end{flushleft}
        \subsection{Il numero di Nepero e}
        \begin{flushleft}
            \textbf{Teorema}: La successione $\{(1+\frac{1}{n})^n\}_{n \geq 1}$ e' convergente
            e il suo limite e' detto: numero di nepero e:
            \begin{equation}
                \lim_{n \to \infty}(1+\frac{1}{n})^n= e
            \end{equation}
            \textbf{Osservazione} Si dimostra che $e \notin \mathbb{Q}$ e il suo valore e'
            \begin{equation}
               e=2.7182812884...
            \end{equation}
            \textbf{Dimostrazione} Verifichiamo la convergenza dimostrando
            \begin{enumerate}
                \item la successione e' strettamente crescente
                \item la successione e' superiormente limitata e dunque il limite eiste e vale
                    \begin{equation}
                        sup(\{(1+\frac{1}{n}^n):n\in \mathbb{N}^+\}) \in \mathbb{R}
                    \end{equation}
            \end{enumerate}
            \begin{enumerate}
                \item Stretta crescenza: $\forall n \geq 1 \quad (1+ \frac{1}{n+1})^{n+1} > (1+\frac{1}{n})^n$
                    \begin{equation}
                        \begin{aligned}
                            (1+ \frac{1}{n+1})^{n+1}=  & \sum_{k=0}^n
                        \begin{pmatrix}
                            n+1 \\
                            k
                        \end{pmatrix} \frac{1}{(n+1)^k}+\frac{1}{(n+1)^{n+1}}\\
                            > & \sum_{k=0}^{n}
                            \begin{pmatrix}
                                n \\
                                k
                            \end{pmatrix} \frac{1}{n^k}=(1+\frac{1}{n})^n
                        \end{aligned}
                    \end{equation}
                \item Limitatezza superiore: $\forall n \geq 2$
                    \begin{equation}
                        \begin{aligned}
                            & (1+\frac{1}{n})^n = \sum_{k=0}^{n}
                            \begin{pmatrix}
                                n \\
                                k
                            \end{pmatrix}\frac{1}{n^k}=1+1+\sum_{k=2}^{n}=\\
                            & =2+\sum_{k=2}^{n}(\frac{1}{k-1}-\frac{1}{k})=2+(1-\frac{1}{2})+(\frac{1}{2}-\frac{1}{3})+....+(\frac{1}{n-1}-\frac{1}{n})\\
                            & = 2+1-\frac{1}{n} < 3
                       \end{aligned}
                    \end{equation}
                    Quindi la successione ha 3 come maggiorante e dunque e' limitata superiormente.
            \end{enumerate}
        \end{flushleft}
        \subsubsection{Esempi}
        \begin{flushleft}
            \begin{itemize}
                \item
                    \begin{equation}
                        \lim_{n \to \infty} (1+\frac{1}{n^2})^{n^2} = e
                    \end{equation}
                    perche' $n^2$ e' una sottosuccessione di $\{n\}_{n\geq 1}$
                \item
                    \begin{equation}
                        \begin{aligned}
                            &  \lim_{n \to \infty} (1-\frac{1}{n})^{n} = \frac{1}{e} \\
                            &  (1-\frac{1}{n})^n=(\frac{n-1}{n})^n=(\frac{n}{n-1})^{-n}=((1+\frac{1}{n-1})^{n-1}*(1+\frac{1}{n-1}))^{-1}
                        \end{aligned}
                    \end{equation}
                \item
                    \begin{equation}
                        \begin{aligned}
                            \lim_{n \to \infty} (1+\frac{1}{n})^{n^2} = +\infty
                            (1+\frac{1}{n})^{n^2}=((1+\frac{1}{n})^{n})^n \geq 2^n \to +\infty
                        \end{aligned}
                    \end{equation}
                    definitivamente perche e $>$ 2
            \end{itemize}
        \end{flushleft}
        \subsubsection{Limite di funzione}
        \begin{flushleft}
            $f:D \to R$, $D\subseteq \mathbb{R}$, $x_o \in \mathbb{R}\cup \{+\infty,-\infty\},l\in \mathbb{R}$ (retta estesa)
        \end{flushleft}
        L'intorno di centro $x_o$ e raggio $r$ e' definito come:$I(x_o,r)$
        \begin{itemize}
            \item $(x_o-r,x_o+r)$ se $x_o \in \mathbb{R}$
            \item $(r,+\infty)$ se $x_o = +\infty$
            \item $(-\infty,-r)$ se $x_o = -\infty$
        \end{itemize}
        Nel caso $x_o \in \mathbb{R}$ si pone:
        \begin{itemize}
            \item $I^+(x_o,r)=(x_o,x_o+r)$ intorno destro
            \item $I^-(x_o,r)=(x_o-r,x_o)$ intorno sinistro
        \end{itemize}
        \subsection{Teorema "Ponte"}
        \begin{flushleft}
            Si chiama cosi perche fa da ponte alle proprieta dei limiti delle successioni e a quelli dei limiti della funzione.
        \end{flushleft}
        \subsubsection{Proprieta' dei limiti di funzione}
        \begin{flushleft}
            Grazie al teorema del ponte valgono le proprieta' analoghe viste per i limiti delle successioni.
        \end{flushleft}
        \subsubsection{Funzioni continue}
        \begin{flushleft}
            Sia $f: D\to R$ e sia $x_o \in D$\\
            Se $x_o$ e' un punto di accumulazione di D e
            \begin{equation}
                \lim_{x \to x_o} f(x)=f(x_o)
            \end{equation}
            Allora si dice f e' continuita' in $x_o$.\\
            F si dice continua in $A\subseteq D$ se f e' continua in ogni punto $x_o \in A$ \\
            \textbf{Osservazione} Le funzioni elementari sono continue nel loro dominio. Per le proprieta dei limiti
            se f e g sono continue allora $f+-g,f*g,f/g$ e $f \circ g$ sono continue nel loro dominio.
        \end{flushleft}
        \subsubsection{Limiti Notevoli}
        \begin{flushleft}
            Per il teorema del "ponte" valgono i seguenti limiti:
            \begin{itemize}
                \item \begin{equation}
                        \lim_{x \to +\infty} \frac{log_a(x)}{x^b}=0
                \end{equation} per $a>0, a \neq 1$ e $b>o$
                \item \begin{equation}
                        \lim_{x \to 0^+} x^b*log_a(x)=0
                \end{equation} $b>o$
                \item \begin{equation}
                        \lim_{x \to +\infty} \frac{x^b}{a^x}=0
                \end{equation} per $a>0$ e $b>o$
                \item \begin{equation}
                        \lim_{x \to 0} \frac{log_a(1+x)}{x}=1
                \end{equation}
                \item \begin{equation}
                        \lim_{x \to 0} \frac{e^x-1}{x}=1
                \end{equation}
                \item \begin{equation}
                        \lim_{x \to 0} (1+\frac{a}{x})^x=e^a
                \end{equation} $a\in \mathbb{R}$
                \item \begin{equation}
                        \lim_{x \to 0} \frac{(1+x)^a-1}{x}=a
                \end{equation} per $a\in \mathbb{R}$
                \item \begin{equation}
                        \lim_{x \to 0} \frac{sin(x)}{x}=1
                \end{equation}
            \end{itemize}
        \end{flushleft}
        \subsection{Confronti tra infinitesimi}
        \begin{flushleft}
            Se:
            \begin{equation}
                \lim_{x \to x_0} \frac{f(x)}{g(x)} =
                \label{eq:}
            \end{equation}
            \begin{enumerate}
                \item  nel caso di 0 diciamo che $x\to x_O$, f(x) e' un
                        infinitesimo di ordine superiore a g(x)
                \item nel caso di $l \neq 0$ diciamo che per $x \to x_0$, f(x) e' un
                        infinitesimo dello stesso ordine di g(x)
                \item nel caso di +- $\infty$ diciamo che per $x \to x_0$, f(x) e' un infinitesimo
                    di ordine inferiore a g(x)
            \end{enumerate}
            Se $\alpha$, per $x\to x_0$, $(x-x_0)^\alpha$ e' un infinitesimo di ordine $\alpha$
        \end{flushleft}
        \subsection{Proprieta delle funzioni continue (1 parte)}
        \begin{flushleft}
            \subsubsection{Teorema degli zeri}
            \begin{flushleft}
                Se f e' una funzione continua in [a,b] e $f(a)*f(b)<0$ allora $\exists x_0 \in (a,b):f(x_0)=0$\\
                Dimostrazione: Caso $f(a)<0$ e $f(b)>0$. Costruiamo ricorsivamente due sucssioni $\{a_n\}_n$ $\{a_n\}_n$
                nel seguente modo: $a_0=a,b_0=b$ e dati $a_0,a_1,....,a_n$ e analogamente per b sia
                \begin{equation}
                    c_n=\frac{a_n+b_n}{2}
                \end{equation}
                Punto medi di $a_n,b_n$ Abbiamo 3 casi possibili:
                \begin{enumerate}
                    \item se $f(c_n)=0$ allora $x_0=c_n$ e abbiamo finito
                    \item se $f(c_n)<0$ allora poniamo $a_{n+1}=c_n$ e $b_{n+1}=b_n$
                    \item se $f(c_n)<0$ allora poniamo $a_{n+1}=a_n$ e $b_{n+1}=c_n$
                \end{enumerate}
                Se il punto 1 non si verifica mai allora sono tali che:
                \begin{equation}
                    \forall n \in \mathbb{N} \quad a_n\leq a_{n+1}<b
                \end{equation}
                $\{a_n\}$ e' crescente ed e' limitata superiormente
                \begin{equation}
                    \forall n \in \mathbb{N} \quad a\leq b_{n+1}<b_n
                \end{equation}
                $\{b_n\}$ e' decrescente ed e' limitata inferiormente e quindi convergono entrambe.
                \begin{equation}
                    \lim_{n\to \infty} a_n=sup{a_n,n\geq 0}=A
                \end{equation}
                \begin{equation}
                    \lim_{n\to \infty} b_n=inf{b_n,n\geq 0}=B
                \end{equation}
                con $a\leq A \leq B \leq b$. Inoltre al passo n-esimo l'intervallo [a,b] e'
                stato diviso a meta' n volte e quindi:
                \begin{equation}
                    B-A \leftarrow b_n-a_n=\frac{b-a}{2^n} \to 0 \to B=A
                \end{equation}
                Chiamiamo questo valore comune $x_0$ e verifichiamo che $f(x_0)=0$\\
                Dato che f e' continua in $x_0 \in [a,b]$,
                \begin{equation}
                    a_n \to x_0 \to f(a_n) \to f(x_0) \leq 0
                \end{equation}
                \begin{equation}
                    b_n \to x_0 \to f(b_n) \to f(x_0) \geq 0
                \end{equation}
                cosi $0\leq f(x_0)\leq 0$ ossia $f(x_0)=0$.
            \end{flushleft}
            \subsubsection{Teorema dei valori intermedi}
            \begin{flushleft}
                Se f e' una funzione continua in [a,b] e $y_0$ e' un valore compreso
                strattemente tra f(a) e f(b) allora $\exists x_0 \in (a,b)$ tale che $f(x_0)=y_0$\\
                \textbf{Osservazione:} Se f e' una funzione continua in un intervallo I allora l'insieme
                immagine $f(I)$ e' ancora un intervallo.
            \end{flushleft}
        \end{flushleft}
        \section{Derivate}
        \begin{flushleft}
         Una funzione e' derivabile se $f: D \to E,x_0 \in D$ e se la formula del rapporto incrementale esiste:
          \begin{equation}
            \exists \lim_{h \to 0} \frac{f(x+h)-f(x_0)}{h} = f'(x_0)=\frac{df}{dx}(x_0) 
          \end{equation}
          "in caso inserire grafico" \\ 
          secante=
          \begin{equation}
            y=(\frac{f(x_0+h)-f(x_0)}{h})(x-x_0)+f(x_0)
          \end{equation}
          al limite abbiamo una retta tangente = $y=f(x_0)(x-x_0)+f(x_0)$
          \textbf{Osservazione:} se f e' derivabile in $x_0$ allora f e' anche continua in $x_0$
          \begin{equation}
           \lim_{x \to x_0} f(x)=f(x_0) 
          \end{equation}
        \end{flushleft}
        \subsection{Derivate delle funzioni elementari}
        \begin{flushleft}
          Ogni derivata della funzione elementare e' stat trovata applicando la formula del rapporto incrementale
          \begin{itemize}
            \item $f(x)=mx+q \to m$
            \item $f(x)=x^b \to bx^{b-1}$
            \item $f(x)=e^x \to e^x$
            \item $f(x)=log(x) \to \frac{1}{x} = ln(x) \to \frac{1}{x}$
            \item $f(x)=sen(x) \to cos(x)$
            \item $f(x)=cos(x) \to -sen(x)$
            \item $f(x)=arcsen(x) \to \frac{1}{\sqrt{1-x^2}}$
            \item $f(x)=arccosen(x) \to -\frac{1}{\sqrt{1-x^2}}$
            \item $f(x)=arctan(x) \to \frac{1}{1-x^2}$
            \item $f(x)=tan(x) \to \frac{1}{cos^2(x)}=1+tg^2(x)$
          \end{itemize}
        \end{flushleft}
        \subsection{Regole di derivazione}
        \begin{flushleft}
         \begin{enumerate}
          \item Linearita': $\forall a,b \in \mathbb{R} \quad (af+bg)'(x)=af'(x)+bg'(x)$ 
          \item Prodotto: $(f*g)'(x)=f(x)'g(x)+f(x)g'(x)$ 
          \item Rapporto: $(\frac{f(x)}{g(x)})'=\frac{f(x)'g(x)-f(x)g'(x)}{g^2(x)}$ 
          \item Composte: $(f\circ g)'(x)=f'(g(x))*g'(x)$ 
         \end{enumerate}
        \end{flushleft}
        \subsection{Punti massimi e minimi}
        \begin{flushleft}
          $f: D \to R, A\subseteq D, x_0 \in A$
          \begin{itemize}
            \item $x_0$ e' un punto di massimo assoluto di f in A se:
              \begin{equation}
                \forall x \in A, f(x) \leq f(x_0) 
              \end{equation}
            \item $x_0$ e' un punto di massimo relativo di f in A se:
              \begin{equation}
                \exists r>0,\forall x \in A \cap I(x_0,r), f(x)\leq f(x_0)
              \end{equation}
            \item analogalmente per il punto minimo assoluto e relativo cambiando il segno delle espressione di sopra indicate
          \end{itemize}
          Questi puntini esistono perche A e' sotto insieme del dominio se noi cambiamo il dominio potrebbero cambiare anche i punti di massimo e 
          di minimo
        \end{flushleft}
        \subsection{Teorema di Fermat}
        \begin{flushleft}
          $f(a,b) \to R$, $x_0$ e' un punto di massimo o minimo relativo in a e b \\ 
          Allora se f e' derivabile in $x_0$ allora $f'(x_0)=0$ questo vuol dire che $x_0$ e' un punto stazionario \\
          \textbf{Dimostraione}: $\exists r>0:(x_0-r,x_0+r) \leq (a,b)$ e $\forall x \quad I(x_0,r) \quad f(x) \leq f(x_0)$
          \begin{equation}
            \frac{f(x_0+h) - f(x_0)}{h}= 
          \end{equation}
          \begin{itemize}
            \item $\frac{\leq 0}{+}\leq 0$ se $h>0 \Rightarrow f(x_0)=0$
            \item $\frac{\leq 0}{-}\geq 0$ se $h<0 \Rightarrow f(x_0)=0$
          \end{itemize}
        \end{flushleft}
        \subsection{Teorema (Bolzano-Weierstrass)}
        \begin{flushleft}
          Se $\{x_n\}_n$ e' una successione limitata allora esiste una sottosucessione $\{x_{n_{k}}\}_k$ convergente
        \end{flushleft}
        \subsection{Teorema Weierstrass}
        \begin{flushleft}
          Se f e' una funzione continua in [a,b] allora $\exists x_{min} \in [a,b],\exists x_{max} \in [a,b]$ 
          \begin{equation}
            \forall [a,b] \quad f(x_{min}) \leq f(x) \leq f(x_{max}) 
          \end{equation}
          Osservazione: $f([a,b])=\{f(x): x \in [a,b] \} = [f(x_{min},f(x_{max})]$
        \end{flushleft}
        \subsubsection{Dimostrazione esistenza di $x_{max}$}
        \begin{flushleft}
          $M=sup\{f(x):x \in [a,b] \} \in \mathbb{R} \cup \{-\infty,+\infty\}$
          \begin{enumerate}
            \item Se $M=+\infty,\forall n \in \mathbb{N}^+,\exists x_n \in [a,b],f(x_n)>n$
            \item Se $M= \mathbb{R},\forall n \in \mathbb{N}^+,\exists x_n \in [a,b]: M-\frac{1}{n} \leq f(x_n)\leq M$
          \end{enumerate}
          In entrambi i casi e' vera:
          \begin{equation}
            \lim_{n\to \infty} f(x_n)=M \quad (f(x_{max})=M) 
          \end{equation}
          La successione $\{x_n\}_n \subseteq [a,b]$ e' quindi limitata e per (Bolzano-Weierstrass) $\exists \{x_{n_{k}}\}$ che converge,
          \begin{equation}
            \exists \lim_{k \to \infty} x_{n_{k}}=x_{n_{0k}} ???
          \end{equation}
          ora rimane da vedere se e' vero $f(x_{max})=M$ dat che f e' continua in $x_{max}$ 
          \begin{equation}
            \lim_{k \to \infty} f(x_{n_{k}}=f(x_{max})
          \end{equation}
          Ora dobbiamo vedere se questo limite e' =M e grazie alla proprieta' dei limiti possiamo dire che se la successione tende ad un valore anche la
          sottosuccessione tendera a  quello stesso valore.
        \end{flushleft}
        \subsection{Teorema di Lagrange o Teorema del valor medio}
        \begin{flushleft}
          $f:[a,b]\to R$, continua nell'intervallo chiuso e derivabile nell intervallo aperto (a,b) \\ 
          Allora $\exists c \in (a,b)$ tali che: (questa formula puo essere interpretata come media)
          \begin{equation}
            \frac{f(b)-f(a)}{b-a}=f(c)
          \end{equation}
        \textbf{Dimostrazione:} 
          \begin{equation}
            f(x)=f(x)-\frac{f(b)-f(a)}{b-a}*(x*a)+f(a) 
          \end{equation}
          Per il teorema di Weierstrass $\exists x_{min},x_{max} \in [a,b]$ punti di minimi e massimo assoluti di I in [a,b]
          \begin{enumerate}
            \item entrambi stiano agli estremi = $x_{min}$ e $x_{max}$ stanno agli estremi a e b ($\{a,b\}$)
              $\Rightarrow h(x_{min})=h(x_{max})=0 \Rightarrow \forall x \in [a,b], h(x)=0$ 
            \item almeno uno tra $x_{min}$ e $x_{max}$ sta in (a,b). Per il teorema di Fermat $h'(x_{max})=0$ oppure $h'(x_{min})=0$
          \end{enumerate}
          Allora sia in 1) che in 2):
          \begin{equation}
            \exists c \in (a,b): h'(c)=0
          \end{equation}
        \end{flushleft}
        \subsection{Teorema crieterio di monotonia}
        \begin{flushleft}
          Sia f derivabile in un Intervallo I allora:
          \begin{enumerate}
            \item $f'(c) \geq \forall x \in I \iff$ f e' crescente in I
            \item $f'(c) \leq \forall x \in I \iff$ f e' decrescente in I
          \end{enumerate}
          \textbf{Osservazione:} $f'(x)=0\quad \forall x \in I \iff $ f e' costante \\
          \textbf{Dimostrazione:}
          \begin{itemize}
            \item ($\Leftarrow$) sia $x\in I$
              \begin{equation}
                \frac{f(x+h)-f(x)}{h}= \frac{\geq 0}{+}\geq 0, h>0 \land \frac{\leq 0}{-}\geq 0, h<0 
              \end{equation}
           \begin{equation}
             \lim_{h \to 0} \frac{f(x+h)-f(x)}{h}= f'(x) \geq 0
           \end{equation}
         \item ($\Rightarrow$) Sappiamo che $\forall x \in I$, $f'(x) \geq 0$ devo dimostrare che :
           \begin{equation}
            \forall x_1,x_2 \in I, x_1<x_2 \Rightarrow f(x_1) < f(x_2) 
           \end{equation}
           \begin{equation}
             \frac{f(x_2)-f(x_1)}{x_2-x_1} = f'(c)\geq 0
           \end{equation}
              $\exists c \in (a,b) \subseteq I$
          \end{itemize}
        \end{flushleft}
        \section{Asintoti}
        \subsection{Asintoti verticali}
        \begin{flushleft}
          $x=x_0$ e' un asintoto verticale di f se
          \begin{equation}
            \lim_{ x \to x_0^+} f(x)= +- \infty \quad \lor \quad \lim_{ x \to x_0^-} f(x)= +- \infty 
          \end{equation}
        \end{flushleft}
        \subsection{Asintoti orizzontali e obliqui}
        \begin{flushleft}
          y=mx+q e' un asintoto di f se
          \begin{equation}
            \lim_{x \to +-\infty} (f(x)-(mx+q))=0
          \end{equation}
          $m=0 \to $ asintoto orizzontale,
          $m\neq  0 \to $ asintoto orizzontale
        \end{flushleft}
        \begin{flushleft}
          \textbf{Osservazione:} Se
          \begin{equation}
            \lim_{x \to +-\infty} \frac{f(x)}{x} =m \in \mathbb{R} \quad \land \quad \lim_{x \to +-\infty} (f(x)-mx)=q \in \mathbb{R}
          \end{equation}
        \end{flushleft}
        \begin{flushleft}
          \textbf{Osservazione:} A,B polinomi
          \begin{equation}
            f(x)=\frac{A(x)}{B(x)}
          \end{equation}
          \begin{itemize}
            \item asintoto obliquo $\to$ grado(A)-grado(b)=1 
            \item asintoto orizzontale $\to$ grado(A)=grado(B)
          \end{itemize}
        \end{flushleft}
        \subsection{Concavita' e convessita'}
        \begin{flushleft}
          Sia $f:D \to \mathbb{R}$, I intervallo $\subseteq D$
        \end{flushleft}
        \begin{flushleft}
          f si dice (strettamente) convessa dove nell'intervallo I: $\forall x_1,x_2 \in I \quad \forall t\in(0,1)$
          \begin{equation}
            f(tx_1+(1-t)x_2) \leq f(x_1)+(1-t)f(x_2)
          \end{equation}
        \end{flushleft}
        \begin{flushleft}
          f si dice (strettamente) concava dove nell'intervallo I: $\forall x_1,x_2 \in I \quad \forall t\in(0,1)$
          \begin{equation}
            f(tx_1+(1-t)x_2) \geq f(x_1)+(1-t)f(x_2)
          \end{equation}
          Oppure se -f e' una funzione convessa
        \end{flushleft}
        \begin{flushleft}
          \textbf{Osservazione:} Ogni funzione retta f(x)=mx+q e' sia concava che convessa
        \end{flushleft}
        \subsubsection{Teorema criterio di convessita'/concavita'}
        \begin{flushleft}
          se f e' derivabile in I intervallo allora 
          \begin{enumerate}
            \item f e' convessa in I $\iff$ f e' crescente in I $\iff$ $f"(x) \geq 0 \quad \forall x \in I$
            \item f e' concava in I $\iff$ f e' decrescente in I $\iff$ $f"(x) \leq 0 \quad \forall x \in I$
          \end{enumerate}
        \end{flushleft}
        \subsection{Punti di flessso}
        \begin{flushleft}
          Sia f continua in $I(x_0)$ e derivabile in $x_0$, $x_0$ si dice punto di flesso:
          \begin{enumerate}
            \item se f e' strettamente convessa in $(x_0-r,x_0)$ e f e' strettamente concava in $(x_0,x_0+r)$
            \item se f e' strettamente concava in $(x_0-r,x_0)$ e f e' strettamente convessa in $(x_0,x_0+r)$
          \end{enumerate}
        \end{flushleft}
        \subsection{Punti di discontinuita'}
        \begin{flushleft}
          $f:D \to \mathbb{R}, x_0 \in D$
        \end{flushleft}
        \begin{enumerate}
          \item Punto di discontinuita' eliminabile. Def:
            \begin{equation}
              \lim_{x \to x_0^+} f(x) = \lim_{x \to x_0^-} f(x)= L \in \mathbb{R}, f(x_0) \neq L
            \end{equation}
          \item Punto di discontinuita' di salto. Def:
            \begin{equation}
              \lim_{x \to x_0^+} f(x) = L^+ \in \mathbb{R},\lim_{x \to x_0^-} f(x) = L^- \in \mathbb{R},L^+ \neq L^-
            \end{equation}
        \end{enumerate}
        \subsection{Punti di non derivabilita'}
        \begin{flushleft}
          Sia f continua in $x_0$
        \end{flushleft}
        \begin{equation}
          \lim_{x \to x_0^+} f'(x)=L^+ \in \mathbb{R} \cup {-\infty,+\infty}
        \end{equation}
        \begin{equation}
          \lim_{x \to x_0^-} f'(x)=L^- \in \mathbb{R} \cup {-\infty,+\infty}
        \end{equation}
        \subsubsection{Punti angolosi,cuspidi e flesso tangente verticale}
        \begin{enumerate}
          \item \textbf{Punto angoloso:}
            \begin{flushleft}
              $L^+ \neq L^-$ e almeno uno deve essere finito
            \end{flushleft}
          \item \textbf{Punto di cuspide:}
            \begin{flushleft}
              $L^+=+\infty,\quad L^-=-\infty$
            \end{flushleft}
          \item \textbf{Punto di flesso tangente verticale:}
            \begin{flushleft}
              $L^+=L^-=+\infty,\quad L^-=L^+=-\infty$
            \end{flushleft}
        \end{enumerate}
        \subsection{Studio di una funzione/grafico}
        \begin{enumerate}
          \item Dominio D
          \item Continuita', simmetrie, segno
          \item Limite degli estremi di D e asintoti
          \item Derivata prima, crescenza/decresecenza, massimi e minimi assoluto e relativi
          \item Derivata seconda, convessita' e concavita', punti di flesso
        \end{enumerate}
        \subsection{Teorema di Cauchy}
        \begin{flushleft}
          Siano f e g continue in un certo intervallo [a,b] e derivabile in (a,b)
        \end{flushleft}
        \begin{flushleft}
          Se $g'(x) \neq 0$ in (a,b) allora $\exists c \in (a,b)$
        \end{flushleft}
        \begin{equation}
          \frac{f(b)-f(a)}{g(b)-g(a)}=\frac{f'(c)}{g'(c)}
        \end{equation}
        \subsection{Teorema Hopital-Bernoulli}
        \begin{flushleft}
          f,g funzioni derivabili in $(x_0,x_0+r)$ con $r>0$
        \end{flushleft}
        \begin{enumerate}
          \item 
            \begin{equation}
              \lim_{x \to x_0^+} f(x)=\lim_{x \to x_0^+} g(x)=0
          \end{equation}
        \item 
          \begin{equation}
            \forall x \in (x_0,x_0+r), g'(x) \neq 0
          \end{equation}
        \item 
          \begin{equation}
            \exists \lim_{x \to x_0}=L \in \mathbb{R} \cup \{ +\infty, -\infty \}
          \end{equation}
          Allora 
            \begin{equation}
              \exists \lim_{x \to x_0^+} \frac{f(x)}{g(x)}=L
            \end{equation}
        \end{enumerate}
        \begin{flushleft}
          Dimostrazione: Estendo f e g in $x_0$ ponendo $f(x_0)=g(x)=0$. Quindi che le f e g estese sono continue in $[x_0,x_0+r]$.
          Considero una successione una successione $\{ x_n \} \subseteq (x_0,x_0+r$ tale che $x_n \to x_0^+$.
        \end{flushleft}
        \begin{flushleft}
          Applico teorema di cauchy in $[x_0,x_n], \exists c_n \in (x_0,x_n)$
        \end{flushleft}
        \begin{equation}
          \frac{f(x_n)}{g(x_n)}=\frac{f(x_n)-f(x_n)}{g(x_n)-g(x_n)}=\frac{f'(c_n)}{g'(c_n)} \to L \square
        \end{equation}
        \subsection{Polinomio Taylor}
        \begin{flushleft}
          f derivabile n-volte in un certo punto $x_0$ allora il polinomio di Taylor di f centrato in $x_0$ di ordine n e'
        \end{flushleft}
        \begin{equation}
          T_{n,x_0}(x)= \sum_{k=0}^n \frac{f^{(k)}(x_0)}{k!}+o((x-x_0)^n)
        \end{equation}
        \subsubsection{Formula di Taylor con resto di Peano}
        \begin{flushleft}
          Se f e' derivabile n-volte in $I(x_0,r)$ con $r>0$ allora:
        \end{flushleft}
        \begin{equation}
          \forall x \in I(x_0,r) \quad f(x)=T_{n,x_0}+o((x-x_0)^n)
        \end{equation}
        \begin{flushleft}
          Dimostrazione:
        \end{flushleft}
        \subsubsection{Principali sviluppi di Taylor per $x \to 0$}
        \begin{itemize}
          \item $e^x=1+x+\frac{x^2}{2!}+\frac{x^3}{3!}+...+\frac{x^n}{n!}+o(x^n)$
          \item $log(1+x)=x-\frac{x^2}{2}+\frac{x^3}{3}-\frac{x^4}{4}+...+(-1)^{n-1}*\frac{x^n}{n}+o(x^n)$
          \item $sen(x)=x-\frac{x^3}{3!}+\frac{x^5}{5!}-\frac{x^7}{7!}+...+(-1)^n*\frac{x^{2n+1}}{(2n+1)!}+o(x^{2n+2})$
          \item $cos(x)=1-\frac{x^2}{2!}+\frac{x^4}{4!}-\frac{x^6}{6!}+...+(-1)^n*\frac{x^{2n}}{(2n)!}+o(x^{2n+1})$
          \item $(1+x)^b=1+bx+\frac{b(b-1)}{2!}x^2+...+\frac{b(b-1)...(b-n+1)}{n!}x^n+o(x^n)$
          \item $arctan(x)=x-\frac{x^3}{3}+\frac{x^5}{5}-\frac{x^7}{7}+...+(-1)^n*\frac{x^{2n+1}}{(2n+1)}+o(x^{2n+2})$
        \end{itemize}
        \subsubsection{Regole per l'o-piccolo}
        \begin{enumerate}
          \item $\forall c \neq 0,\quad  c*o(x^n)=o(c*x^n)=o(x^n)$
          \item $b>a \quad c*o(x^b)-o(x^a)=o(x^a)$
          \item $b>a \quad o(x^b)-o(x^a)=o(x^a)$
          \item $o(x^a)*o(x^b)=o(x^{a+b})$
          \item $x^a*o(x^b)=o(x^{a+b})$
          \item ...
        \end{enumerate}
        \section{Calcolo integrale}
        \begin{flushleft}
          
        \end{flushleft}
        \subsection{Teorema}
        \begin{flushleft}
          Se f e' continua in [a,b] allora f e' integrabile in [a,b]
        \end{flushleft}
        \begin{flushleft}
          \textbf{Osservazioni}:
          \begin{enumerate}
            \item Se f e'integrabile in [a,b] allora anche la funzione che ottengo da f modificandola in un numero finito di punti
              e' integrabile e ha lo stesso integrale. Questo vuol dire posso inserire finiti segmenti e punti ma l area rimarra sempre la stessa
            \item 
              \begin{equation}
                \int_a^b f(x)dx = -\int_b^a f(x)dx \quad \quad  \int_a^a f(x)dx=0
              \end{equation}
            \item Se f e' integrabile in [a,b] e $c \in $[a,b]
              \begin{equation}
                \int_a^b f(x)dx = \int_a^c f(x)dx +\int_c^b f(x)dx 
              \end{equation}
            \item Linearita'
              \begin{equation}
                \int_a^b (\alpha f(x)+ \beta f(x))dx=\alpha \int_a^b f(x)+ \beta \int_a^b f(x))dx
              \end{equation}
              \begin{equation}
                \int_a^b (f(x)g(x))dx \neq \int_a^b f(x)dx * \int_a^b g(x)dx 
              \end{equation}
            \item Monotonia
              Se $f(x) \leq g(x)$ in [a,b]
              \begin{equation}
                \int_a^b f(x) \leq \int_a^b g(x)
              \end{equation}
              Inoltre
              \begin{equation}
                -\mid f(x) \mid \leq f(x) \leq \mid f(x) \mid
              \end{equation}
              Questo implica anche:
              \begin{equation}
                -\int_a^b \mid f(x) \mid \leq \int_a^b f(x) \leq \int_a^b \mid f(x) \mid 
              \end{equation}
              Pero non vale l uguaglianza di:
              \begin{equation}
                \mid \int_a^b f(x)dx \mid \leq \int_a^b f(x)dx
              \end{equation}
            \item Se f e' continua 
              \begin{equation}
                \int_a^b f(x)dx = \lim_{n \to +\infty} \frac{1}{n} \sum_{k=1}^n f(a+ \frac{k}{n}(b-a))
              \end{equation}
          \end{enumerate}
        \end{flushleft}
        \subsection{Tecniche di Integrazione}
          \begin{enumerate}
            \item Integrazione per sostituzione
              \begin{flushleft}
                Se F=f allora
              \end{flushleft}
              \begin{equation}
                \begin{split}
                  \int f(g(x))g'(x)dx= & \int f(t) dt=F(t)+c=F(g(x))+c \\ 
                  & t=g(x) \\ 
                  & dt=g'(x)dx
                \end{split}
              \end{equation}
            \item Integrazione per parti
                \begin{equation}
                  \int f(x)g'(x)dx=f(x)g(x)-\int g(x)*f'(x)dx
                \end{equation}
          \end{enumerate}
          \subsection{Integrazioni Razionali $\frac{P}{Q}$}
          \begin{enumerate}
            \item Se deg(P) $\geq$ deg(Q) fare la divisione
              \begin{equation}
                \frac{P}{Q}=S+\frac{R}{Q}
              \end{equation}
              Dove S e' il quoziente e R e' il resto della divisione
            \item Fattorizzare Q
              \begin{equation}
                Q=\Pi_{k=1}^n(x-a_k)*\Pi_{k=1}^m(x^2+b_k+c_k)
              \end{equation}
            \item Scrivere $\frac{P}{Q}$ come combinazioni lineare di fratti semplici
              \begin{itemize}
                \item 
                  \begin{equation*}
                    \frac{1}{(x-a_k)^j} \quad j=1,...,n_k
                  \end{equation*}
                \item 
                  \begin{equation*}
                    \frac{1}{(x^2+b_kx+c_k)^j}+\frac{1}{(x^2+b_kx+c_k)^j} \quad j=1,...,n_k
                  \end{equation*}
              \end{itemize}
              \begin{flushleft}
                ESEMPIO:
              \end{flushleft}
              \begin{equation*}
                \frac{R(x)}{(x+1)^3(x+2)(x^2+1)^2}=\frac{C_1}{x+1}+\frac{C_1}{(x+1)^2}+\frac{C_3}{(x+1)^3}+\frac{C_4}{x+2}+\frac{C_5x+C_6}{x^2+1}+\frac{C_7x+C_8}{(x^2+1)^2}
              \end{equation*}
            \item Usa la linearita' e integra i tratti semplici
              \begin{flushleft}
                Integrali dei tratti semplici
              \end{flushleft}
              \begin{itemize}
                \item j=1 
                  \begin{equation*}
                    \frac{1}{x-a} \to^{\int} lg \mid x-a \mid +c
                  \end{equation*}
                \item j$\geq$1 
                  \begin{equation*}
                    \frac{1}{(x-a)^j} \to^{\int} \frac{(x-a)^{-j+1}}{-j+1}
                  \end{equation*}
                \item continuare con gli appunti di del prof
              \end{itemize}
          \end{enumerate}
          \subsection{Integrali impropri}
          \begin{flushleft}
            $f:[a,b) \lor (a,b] \to \mathbb{R}$
          \end{flushleft}
          \begin{equation*}
            \lim_{t \to b} \int_a^t f(x)dx=\int^b_a f(x)dx=
          \end{equation*}
          \begin{itemize}
            \item $=L\in \mathbb{R} \to $ l'integrale e' convergente
            \item $=+\infty,-\infty \to $ l'integrale e' divergente 
            \item $=\nexists\to $ l'integrale e' indeterminato
          \end{itemize}
          \begin{flushleft}
            $f:(a,b) \to \mathbb{R}$
          \end{flushleft}
          \begin{equation*}
            \int_a^b f(x)dx=\int^c_a f(x)dx + \int^b_c f(x)dx
          \end{equation*}
          \begin{flushleft}
            convergente $\iff$ convergente $\land$ convergente 
            divergente $\iff$ divergente $\land$ divergente
          \end{flushleft}
          \begin{flushleft}
            Osservazione: se $f\geq 0$ in [a,b) allora la funzione
          \end{flushleft}
          \begin{equation*}
            F(t)=\int^t_a f(x)dx \quad \text{e' crescente in [a,b)}
          \end{equation*}
          \begin{flushleft}
            se $t_1<t_2$
          \end{flushleft}
          \begin{equation*}
            F(t_1)-F(t_2)=\int^{t_2}_{t_1}dx \geq 0
          \end{equation*}
          \begin{equation*}
            \int^b_a f(x)dx=\lim_{t \to b^-} F(t)= \text{esiste $\in \mathbb{R} \cup \{+\infty\}$}
          \end{equation*}
          \subsubsection{Teorema del confronto per integrali impropri}
          \begin{flushleft}
            Siano f,g definite in [a,b) e $0\leq f(x)\leq g(x) \quad \forall x\in [a,b)$ allora
          \end{flushleft}
          \begin{enumerate}
            \item Se $\int^b_a g(x)dx$ e' convergente $\to \int^b_a f(x)$ e' convergente
            \item Se $\int^b_a f(x)dx$ e' divergente $\to \int^b_a f(x)$ e' divergeente
          \end{enumerate}
          \begin{flushleft}
            Dimostrazione: Consideriamo
          \end{flushleft}
          \begin{equation*}
            F(t)=\int^t_a f(x)dx,G(t)=\int^t_a g(x)dx
          \end{equation*}
          \begin{equation*}
            G(t)-F(t)=\int^t_a g(x)-f(x)dx \geq 0
          \end{equation*}
          \begin{equation*}
            \int^b_a f(x)dx=\lim_{t \to b^-}F(t) \leq \lim_{t \to b^-}G(t) = \int^b_a g(x)dx
          \end{equation*}
          \subsubsection{Integrali "Notevoli"}
          \begin{equation*}
            \int^{+\infty}_2 \frac{1}{x^{\alpha}(lg(x))^{\beta}}dx=
          \end{equation*}
          \begin{itemize}
            \item converge se $\alpha>1$ oppure $\alpha=1$ e $\beta>1$
            \item diverge altrimenti
          \end{itemize}
          \begin{equation*}
            \int^{\frac{1}{2}}_0 \frac{1}{x^{\alpha}(lg(x))^{\beta}}dx=
          \end{equation*}
          \begin{itemize}
            \item converge se $\alpha<1$ oppure $\alpha=0$ e $\beta>1$
            \item diverge altrimenti
          \end{itemize}
          \subsubsection{Teorema del confronto asintotico}
          \begin{flushleft}
            Siano f,g in [a,b) e $f\geq 0,g>0$ in [a,b) se
          \end{flushleft}
          \begin{equation*}
            \lim_{x\to b^-}\frac{f(x)}{g(x)}=(0,+\infty), \quad f(x) \sim L*g(x) \quad \text{"$\sim$" = equivalenza asintotica}
          \end{equation*}
          \begin{flushleft}
            E.g
          \end{flushleft}
          \begin{equation*}
            \int^{+\infty}_0 \frac{3x^2+1}{4x^3+x+1}dx \quad \text{e' convergente?}
          \end{equation*}
          \begin{equation*}
            \frac{3x^2+1}{4x^3+x+1}=\lim_{x \to +\infty} 
            \frac{x^2(3+\frac{1}{x^2})}{x^3(4+\frac{1}{x^2}+1)} \sim \frac{1}{x}*\frac{3}{4}
          \end{equation*}
          \begin{flushleft}
            Quindi no e' divergente
          \end{flushleft}
          \section{Serie Numeriche}
          \begin{equation*}
            \sum_{k=k_0}^{\infty} A_k= \lim_{n \to \infty} A_k=
          \end{equation*}
          \begin{itemize}
            \item $L\in \mathbb{R}$
            \item $+\infty,-\infty$
            \item $\not \exists$
          \end{itemize}
          \begin{flushleft}
            La seconda parte della equazione iniziale dove c'e' il limite. Quella somma li si chiama somma parziale. Praticamente
            e come se facessimo:
          \end{flushleft}
          \begin{equation*}
            \int^{+\infty}_a f(x)dx=\lim_{t \to +\infty} \int^t_a f(x)dx
          \end{equation*}
          \begin{flushleft}
           Adesso vediamo 3 esempi di serie numerica, 1 per ogni caso indicato sopra: convergente, divergente, indeterminata. 
          \end{flushleft}
          E.g
          \begin{equation*}
            0.1\bar=\sum^{\infty}_{k=1} \frac{1}{10^k}=\frac{1}{9} \quad \Rightarrow \text{converge}
          \end{equation*}
          \begin{equation*}
            \sum^{\infty}_{k=1} 1 \quad \Rightarrow \text{diverge}
          \end{equation*}
          \begin{equation*}
            \sum^{\infty}_{k=1} (-1)^k=-1+1-1+1-1.... \quad \Rightarrow \not \exists
          \end{equation*}
          \subsection{Serie geometrica}
          \begin{equation*}
            \sum^{\infty}_{k=1} x^k
          \end{equation*}
          \begin{itemize}
            \item $+\infty$ se $x=1$
            \item $?$ se $x\neq1$
              \begin{itemize}
                \item $0$ se $\mid x \mid <1$
                \item $+\infty$ se $x>1$
                \item $\not \exists$ se $x \leq 1$
              \end{itemize}
          \end{itemize}
          \begin{flushleft}
            Ora calcoliamo la serie geometrica per vedere che valori assume al variare di x
          \end{flushleft}
          \begin{equation*}
            \sum^{\infty}_{k=1} x^k=\begin{cases}
              \frac{1}{1-x} \quad  \text{se} \quad \mid x \mid <1
              +\infty \quad  \text{se} \quad x >1
              \not \exists \quad  \text{se} \quad x \leq -1
            \end{cases}
          \end{equation*}
          \subsection{Teorema Condizione necessaria di convergenza}
          \begin{equation*}
            \text{Se} \quad  \sum^{\infty}_{k=0} \quad \text{e' convergente allora} \quad \lim_{k\to \infty} A_k=0
          \end{equation*}
          \begin{flushleft}
            Dimostrazione
          \end{flushleft}
          \begin{equation*}
            \lim_{n\to +\infty} S_n=L \in \mathbb{R}
          \end{equation*}
          \begin{equation*}
            0=L-L \Leftarrow S_n-S_{n-1}=\sum^n_{k=0} a_n - \sum^n_{k=0} a_{n-1}=a_n
          \end{equation*}
          \subsection{Serie armonica}
          \begin{equation*}
            \sum^{\infty}_{k=1} \frac{1}{k} = \infty
          \end{equation*}
          \subsection{Teorema Criterio del confronto}
          \begin{flushleft}
            $\{ a_k \},\{b_k\}$, tali che $0 \leq a_k \leq b_k \quad \forall k \geq n$
          \end{flushleft}
          \begin{enumerate}
            \item 
              \begin{equation*}
                \sum^{\infty}_{k=0} b_k= \quad \text{convergente} \quad \Rightarrow \sum^{\infty}_{k=0} a_k \quad \text{e' convergente}
              \end{equation*}
            \item 
              \begin{equation*}
                \sum^{\infty}_{k=0}a_k=+\infty \Rightarrow \sum^{\infty}_{k=0} b_k = +\infty
              \end{equation*}
          \end{enumerate}
          \subsection{Teorema del confronto asintotico}
          \begin{flushleft}
            $\{ a_k \},\{b_k\}$, tali che $0 \leq a_k , b_k >0$ definitivamente
          \end{flushleft}
          \begin{equation*}
            \lim_{k\to \infty} \frac{a_k}{b_k} = L \in (0,+\infty) \quad a_k \Tilde 2b_k
          \end{equation*}
          allora
          \begin{equation*}
            \sum^{\infty}_{k=0} a_k \quad \text{converge} \quad \iff \sum^{\infty}_{k=0} b_k \quad \text{converge}
          \end{equation*}
          \subsection{Teorema del confronto integrale}
          \begin{flushleft}
            Sia f una funzione decrescente e $\geq 0$ in $[1,+\infty)$ e sia $a_k=f(k)$. Allora
          \end{flushleft}
          \begin{equation*}
            \sum^{\infty}_{k=1} a_k \quad \text{converge} \quad \iff \int^{+\infty}_1 f(x)dx \quad \text{converge}
          \end{equation*}
          \subsection{Teorema criterio della radice}
          \begin{equation*}
            \text{Se} \quad \lim_{k\to +\infty} \sqrt[k]{a_k}=L \in [0,+\infty]
          \end{equation*}
          allora:
          \begin{itemize}
            \item $L<1 \Rightarrow \sum^{\infty}_{k=0} a_k$ e' convergente 
            \item $L<1 \Rightarrow \sum^{\infty}_{k=0} a_k=+\infty$
          \end{itemize}
          \subsection{Teorema criterio del rapporto}
          \begin{equation*}
            \text{Se} \quad \lim_{k\to +\infty} \frac{a_{k+1}}{a_k}=L \in [0,+\infty]
          \end{equation*}
          allora
          \begin{itemize}
            \item $L<1 \Rightarrow$ la serie e' convergente 
            \item $L>1 \Rightarrow$ la serie  = $+\infty$
          \end{itemize}
          \begin{flushleft}
            Se L=1 in entrambi i criteri, il risultato e' inconcludenti
          \end{flushleft}
          \begin{equation*} 
            \text{Oss:} \quad \sum^{\infty}_{k=0} \mid a_k \mid \quad \text{converge allora anche} \sum^{\infty}_{k=0} a_k \quad \text{converge}
          \end{equation*}
          \subsection{Teorema criterio di Leibniz}
          \begin{enumerate}
            \item $a_k \geq a_{k+1} \quad \forall k>0$
            \item $\lim_{k\to \infty} a_k=0$
          \end{enumerate}
          \begin{flushleft}
           Se questi 2 criteri sono verificati allora 
          \end{flushleft}
          \begin{equation*}
            \sum^{\infty}_{k=0} (-1)^ka_k \quad \text{e' convergente}
          \end{equation*}
          \begin{flushleft}
            Dimostrazione inserire immagine
          \end{flushleft}
          \section{Numeri complessi}
          \begin{flushleft}
            $\mathbb{C}=\{x+iy:x,y\in \mathbb{R}\} \supseteq R=\{x+i*0: x\in \mathbb{R} \}$
          \end{flushleft}
          \begin{flushleft}
            z=x+iy \\
            $\mid z \mid$=module di z= distanza dall origine al punto z = $\sqrt{x^2+y^2}$
          \end{flushleft}
          \begin{flushleft}
            Re(z)=x parte reale di z \\
            Im(z)=y la parte immaginaria di z
          \end{flushleft}
          \begin{flushleft}
            $\bar{z}$=coniugato di z=x-iy=cambiare il segno della parte immaginaria
          \end{flushleft}
          \subsection{Operazioni su $\mathbb{C}$: somma e prodotto}
          \begin{flushleft}
            $z_1=x_1+iy_1,z_2=x_2+iy_2$ 
          \end{flushleft}
          \subsection*{Somma:}
          \begin{itemize}
            \item $z_1+z_2=(x_1+x_2)+i(y_1+y_2)$
            \item il neutro, 0=0+0*i, l opposto di z
          \end{itemize}
          \subsection*{Prodotto:}
          \begin{itemize}
            \item $z_1*z_2=(x_1+iy_1)(x_1+iy_1)=x_1x_2+ix_1y_2+ix_2y_1+i^2y_1y_2=$ \\
              $=(x_1x_2-y_1y_2)+i(x_1y_2+x_2y_1)$
            \item elemento neutro=1=1+i*0
            \item reciproco di z=x+iy $\neq$0
          \end{itemize}
          \begin{equation*}
            \frac{1}{z}=\frac{1}{x+iy}*\frac{x-iy}{x-iy}=\frac{x-iy}{x^2+y^2}=\frac{x}{x^2+y^2}*\frac{iy}{x^2+iy^2}
          \end{equation*}
          \begin{equation*}
            \frac{1}{z}=\frac{\bar{z}}{\mid z \mid ^2}
          \end{equation*}
          \begin{equation*}
            z*\bar{z}=\mid z \mid ^2 \quad \text{Sviluppando i calcoli}
          \end{equation*}
          \subsection{Proprieta'}
          \begin{enumerate}
            \item $\bar{\bar{z}}=z,\bar{z+w}=\bar{z}+\bar{w},\bar{zw}=\bar{z}*\bar{w},\mid \bar{z}\mid = \mid z \mid$
            \item Re(z)=$\frac{z+\bar{z}}{2}$, Im(z)=$\frac{z-\bar{z}}{2i}$
            \item $\mid zw \mid=\mid z \mid * \mid w \mid$
            \item $\mid z+w \mid \leq \mid z \mid + \mid w \mid$
          \end{enumerate}
          \subsection*{Indice ciclico della i}
          \begin{equation*}
            i^n=
            \begin{cases}
              1 \quad \text{se} \quad n \equiv 0 \quad mod(4)\\
              i \quad \text{se} \quad n \equiv 1 \quad mod(4)\\
              -1 \quad \text{se} \quad n \equiv 2 \quad mod(4)\\
              -i \quad \text{se} \quad n \equiv 3 \quad mod(4)
            \end{cases}
          \end{equation*}
          \begin{flushleft}
            Oss: se $z\neq 0$ allora iz e' il numero che si ottiene ruotando z di 90 gradi in senso antiorario sul grafico dei numeri complessi
          \end{flushleft}
          \subsection*{Perche esistono i numeri complessi}
          \begin{flushleft}
            In $\mathbb{R}$ l'equazione
          \end{flushleft}
          \begin{equation*}
            x^2=a \quad \text{a $\in \mathbb{R}$}
          \end{equation*}
          Se
          \begin{enumerate}
            \item Se $a>0$ ci sono 2 soluzioni
            \item Se $a=0$ c'e' solo una soluzione $\to x=0$ (doppia soluzione)
            \item Se $a<0$ non ci sono soluzioni
          \end{enumerate}
          \begin{flushleft}
            In $\mathbb{C}$ l'equazione
          \end{flushleft}
          \begin{equation*}
            x^2=a \quad \text{a $\in \mathbb{R}$}
          \end{equation*}
          \begin{flushleft}
            Si risolve cosi, $z=x+iy$, $z^2=x^2+iyx+i^2y^2=(x^2-y^2)+i(2xy)=a$
          \end{flushleft}
          \begin{equation*}
            (x^2-y^2)+i(2xy)=a+0 \to 
            \begin{cases}
              x^2-y^2=a  \\
              2xy=0
            \end{cases} \to 
            \begin{cases}
              y^2=-a \\
              x=0
            \end{cases} \cup 
            \begin{cases}
              x^2=a \\
              y=0
            \end{cases}
          \end{equation*}
          \begin{flushleft}
            L'equazione iniziale possiamo vedere che ce uno 0 e la a questo perche la a va messa in sistema con la parte reale della equazione e lo 0
            rappresenta la parte immaginaria dell equazione. Dopo averle messe in sistema risolvendoli vediamo che l ultimo sistema rappresenta
            la soluzione se $a\geq 0$ e quello prima rappresenta la soluzione se $a\leq 0$ e in questa soluzione toccaa usare l ausilio del simbolo i dei numeri complessi
          \end{flushleft}
          \begin{flushleft}
            pupup
            
          \end{flushleft}
      \end{document}
