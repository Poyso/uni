\documentclass{article}
\begin{document}
    \section{Come dimostrare}
    Diagramma di venn serve per l intuizione: per dirti se e vero o sbagliato:
    \subsection{Se e' vero}
    \begin{enumerate}
        \item Tabella di verita
        \item Ragionamento
    \end{enumerate}
    \subsection{Se e' falso}
    \begin{enumerate}
        \item Creare un esempio mettendo numeri negli appositi insiemie e di conseguenza verificare se le 2 espressioni riportano lo stesso numero
    \end{enumerate}
   \section*{Proprieta 1.2.4}
   A,B,C insiemi allora $Ax(B\cap C)=(AxB)\cap(AxC)$\\
   Dimostrazione: Sia $(x,y)\in Ax(B \cap C)\to x\in A \quad e \quad y\in B \quad y\in C$
   che equivale a dire $(x,y)\in AxB \quad e \quad (x,y)\in AxC \to$\\
   $(x,y)\in (AxB)\cup(AxC)$\\
   Viceversa dobbiamo dimostrare $x\in (AxB)\cap(AxC)$\\
   $(x,y)\in Axb$ e $(x,y)\in AxB$\\
   $x\in A$ e $x\in B$ e $y\in C \to (x,y)\in Ax(B \cap C)$\\
   Oss: $x\in A\cup B$ = $x\in A$ e $x \in B$\\
   $(x,y)\in A x B \to x\in A $ e $y\in B$
   \section*{Applicazioni tra insiemi}
   A,B insiemi un applicazione da A in B e' una legge che associa ogni elemento
   di A uno e un solo elemento di B\\
   Def: un applicazione (o funzione, o mappa) da A in B e' un sottoinsieme:
   $F \subseteq AxB \quad : \quad \forall a \in A \to \exists! b\in B \quad : \quad (a,b)$
   si scrive $f(a)=b$, $F:A\to B$\\
   Def: la composizione di g con f e' la funzione $(g \circ f)(a)=g(f(a))$=parto A e arrivo a C applicando le 2 funzioni g e f
   \subsection{Proprieta}
   Def: A,B,C insiemi e $F: A\to B$ e $F:B\to C$
   \begin{enumerate}
    \item F e' inniettiva e g e' inniettiva $g \circ f $ e' inniettiva
    \item F e' surriettiva e g e' surriettiva $g \circ f $ e' surriettiva
    \item F e' biunivoca e g e' biunivoca $g \circ f $ e' biunivoca
   \end{enumerate}
   Dimostrazione:
   \begin{enumerate}
    \item $a_1,a_2 \in A, a_1\neq a_2 \to f(a_1)\neq f(a_2)\to (g\circ f)(a_1)\neq (g\circ f)(a_2)$
    \item $c\in C \to \exists b\in B : g(b)=c \to \exists a\in A: f(a)=b\to (g\circ f)(a)=c$
   \end{enumerate}
   \begin{itemize}
    \item E.g $F=\{ (1,3),(2,4),(4,2)\}$ non e' una funzione perche non ce una corrispondenza per il 3
    \item E.g $A=[4],B=[5] \quad F=\{(1,3),(1,4)...\}$ non e' una funzione perche ci sono due corrispondenze per 1
    \item E.g $F=\{(1.4),(2,5),(3,4),(4,2)\}$ e' una funzione
   \end{itemize}
   \section*{Matematica 0}
   \subsection{Frazioni}
    $\frac{a}{b}+ \frac{c}{d}=\frac{ad+bc}{bd}$
    \subsection{Proprieta}
    -$ln(ab)=ln(a)+ln(b)$\\
    -$ln(1)=0$\\
    $e\in R$ e' l unico numero tale che $ln(e)=1$\\
    sia $x\to e^x$ la funzione inversa di $x\to ln(x)$\\
    \textbf{Proprieta}
    \begin{itemize}
        \item $e^x=y \iff x=ln(y)$
        \item $e^{a+b}=e^a*e^b$
    \end{itemize}
    \subsection{Logaritmi}
    $x>0,b>0,b\neq 1$\\
    $log_b(x)=\frac{ln(x)}{ln(b)} \quad \quad log_b(b)=1$
    \subsection{Esponenziali}
    Se $a>0,a\in R$
    \begin{itemize}
        \item $a^x=e^{x*ln(a)}$
        \item $y=a^x \iff x=\frac{ln(y)}{ln(a)}=log_a(y)$
    \end{itemize}
\end{document}
