\documentclass{article}
\usepackage{amsmath}
\usepackage{amssymb}
\usepackage{amsfonts}
\usepackage{multirow}
\title{Matematica Discreta}
\begin{document}
    \section{Prima parte}
    \begin{flushleft}

    \end{flushleft}
    \section{Numeri}
    \subsection{Principio di Induzione}
    \begin{flushleft}
        In matematica esistono 3 diversi modi per dimostrare
        $A\to B$
        \begin{enumerate}
            \item Dimostrazione Diretta
            \item Dimostrazione per assurdo
            \item Principio di induzione matematica
        \end{enumerate}
        \subsubsection{Esempio}
        \begin{equation}
            \sum_{i=0}^n i = \frac{(n+1)n}{2}
            \label{eq:1 }
        \end{equation}
        \textbf{Passo Base}\\
        \begin{equation}
            \sum_{i=0}^1 i = \frac{(1+1)1}{2} = 1
            \label{eq:2 }
        \end{equation}
        \textbf{Passo Induttivo}\\
        Supponiamo vera P(m) per $\forall m<n$
        \begin{equation}
            \sum_{i=0}^n i = \sum_{i=0}^{n-1} i+n= \frac{(n+1)n}{2} +n = n(\frac{n-1}{2} +n)=\frac{(n+1)n}{2}
            \label{eq: 3}
        \end{equation}
        Pertanto n e' vera $\forall n \in \mathbb{P}$
    \end{flushleft}
    \subsection{Il principio del buon ordinamentoa (WOP)}
    \begin{flushleft}
            $S \subseteq \mathbb{Z}$ allora $\rightarrow \exists m\in S$ tale che $m\leq x$ per $\forall x\in S$
    \end{flushleft}
    \begin{flushleft}
        \textbf{Osservazione:} Falso per $\mathbb{Z} (S = \{-1,-2,-3\}$ non ha il minimo
    \end{flushleft}
    \subsection{Numeri Positivi,Interi e razionali}
    \begin{flushleft}
       Questi Insiemi numerici li abbiamo gia definiti all inizio degli appunti.
        Equivalenza di $\mathbb{Z} x (\mathbb{Z} / \{0\})$
    \end{flushleft}
    \subsection{Numeri Reali}
    \begin{flushleft}
    Si veda il corso di analisi. Intuitivamente i numeri reali sono le lunghezze dei segmenti.
    \end{flushleft}
    \subsection{Numeri Primi e composti}
    \begin{flushleft}
        Sono $a,b\in \mathbb{P}$\\
        Definizione: si dice che a divide b ( o che b e' multiplo di a) se $\exists k \in \mathbb{Z}$ tale che $b=k*a$
        \textbf{Osservazioni}
        \begin{itemize}
            \item $a|b \rightarrow a\leq b$
            \item $a|b \land b|c \rightarrow a|c$
            \item $a|b \land a|c \rightarrow a|(xb+yc) \forall x,y\in \mathbb{Z}$
        \end{itemize}
        Definizione: sia $a\in \mathbb{P}$, a si dice primo se $b|a \rightarrow b=1 \lor b=a \land a\geq 2$ altrimeenti a si dice composto \\
        Siano $a,b\in \mathbb{P}$ si dicono coprimi (o primi tra loro) se:
        \begin{equation}
            c|a \land c|b \rightarrow c=+-1
        \end{equation}
        \textbf{Osservazione:} $p_1$ e $p_2$ primi, $p_1 \neq p_2 \rightarrow$ $p_1$ e $p_2$ sono coprimi\\
       Definzione: a si dice perfetto se a e' uguale alla somma dei suoi divisori.
       \begin{itemize}
           \item 6 e' perfetto (1+2+3=6)
           \item 8 non e' perfetto (1+2+4 $\neq$ 8)
       \end{itemize}
    \end{flushleft}
    \subsubsection{Teorema}
    \begin{flushleft}
        Sia $n\in \mathbb{P}$ allora o n e' primo o n e' prodotto di primi. \\
        \textbf{Dimostrazione} Induzione completa, se n=2 e' un numero primo supponiamo che il Teorema
        sia vero per $\forall m \in \mathbb{P}$ $2 \leq m \leq n$
        \begin{itemize}
            \item Se n e' primo ok
            \item se non e' primo $\rightarrow \exists a,b \in \mathbb{P}$ tali che n=n*b
                ma per induzione , a e b sono o primi o prodotti di primi $\rightarrow $ n allora e' prodotto di primi
        \end{itemize}
    \end{flushleft}
    \subsubsection{Teorema}
    \begin{flushleft}
      Ci sono infiniti numeri primi: \textbf{Dimostrazione} per assurdo. Supponiamo che ci siano un numero finito di numeri primi $\{p_1,p_2,...,p_n\}$ 
      \begin{equation}
       N = p_1*p_2*...*p_n+1 
      \end{equation}
      Allora come dimostrato in 2.5.1 N e' prodotto di numeri primi per tanto $\exists q \in \mathbb{P}$, q e' primo, tale che q|N. \\
      Allora $q\notin \{p_1,...,p_n\}$. Se $q=p_1$ $\to$ $q|p_1$ e $q|N$ $\to$ $q|p_1,....p_n$ e q|N $\to$ $q|(N-p_1*p_2*...*p_n)$ $\to$ q|1 cioe' $q\leq1$ e quindi assurdo
      se $q\geq2$ similmente avremmo che: $q=p_2,q=p_3$, etc.
    \end{flushleft}
    \subsubsection{Teorema (Hadamard-De la valle poussin}
    \begin{flushleft}
      \begin{equation}
        \lim_{n \to \infty} \frac{\pi(n)}{\frac{n}{ln(n)}} =1
      \end{equation}
    \end{flushleft}
    \subsection{Algoritmo Euclide}
    \begin{flushleft}
      Siano $a,b\in \mathbb{P}$ \\ 
      \textbf{Definizione}: il massimo comune divisore di a e b, scritto: MCD(a,b) (o GCD(a,b)) e':
      \begin{equation}
        c = max\{ n \in \mathbb{P}: n|a \land n|b \} 
      \end{equation}
      Come calcolare MCD(a,b)? \\ 
    \end{flushleft}
    \subsubsection{Lemma}
    \begin{flushleft}
      Siano $a,b \in \mathbb{P}$ , $a\geq b$. Allora $\exists q,r \in \mathbb{Z}$ tali che a=bq+r e $0\leq r <b$ \\ 
      \textbf{Dimostrazione:} e' nota
      \textbf{Agoritmo Euclideo}\\       
      Siano $a,b \in \mathbb{P}$, $a\geq b$. Allora $\exists q,r \in \mathbb{Z}$ tali che a=bq+r e $0\leq r <b$ \\
      Se $r=0$ $\to$ MCD(a,b)=b \\ 
      Se $r>0$ $\to$ $\exists  q,r \in \mathbb{Z}$ tali che $b=q_1*r+r_1$ e $0\leq r_1 <r$
      Se $r_1=0$ $\to$ MCD(a,b)=r \\ 
      Se $r_1>0$ $\to$ $\exists  q_2,r_2 \in \mathbb{Z}$ tali che $b=q_2*r_1+r_2$ e $0\leq r_2 <r_1$
      Se $r_2=0$ $\to$ MCD(a,b)=$r_1$ \\ 
      Se $r_2>0$ $\to$ $\exists  q_3,r_3 \in \mathbb{Z}$ tali che $b=q_3*r_2+r_3$ e $0\leq r_3 <r_2$ etc... \\ 
      In questo modo si ottiene una sequenza di numeri $b>r>r_1>r_2>...\geq 0$ \\ 
      $\exists k \in \mathbb{P}$ tale che $r_k=0$. L'algoritmo termina allora MCD=$r_{k-1}$ \\ 
      \textbf{Perche' funziona?} \\ 
      L'Algoritmo di Euclide produce due sequenze di numeri $r_1,r_2,r_3...$ e $q_1,q_2,q_3...$ tali che
        \begin{equation}
           a = bq +r  
           b= q_1*r+r_1  
           r_i = q_{i+2}*r_{i+1} + r_{i+2} % da allineare
        \end{equation}
        Sia $k\in \mathbb{P}$ tali che $r_k=0$. Ma
        \begin{equation}
          r_{k-2}=q_k*r_{k-1} + r_k \to r_{k-1}|r_{k-2} 
        \end{equation}
        \begin{equation}
          r_{k-3}=q_{k-1}*r_{k-2} + r_{k-1} \to r_{k-1}|r_{k-3} 
        \end{equation}
        \begin{equation}
          b=q*r+r_1 \to r_{k-1}|b \quad \quad a=b*q+r\to r_{k-1}|a 
        \end{equation}
        Quindi $r_{k-1}$ e' un divisore comune di a e b
        \\
        Sono $a,b \in \mathbb{P}, a\geq b$ L'algoritmo di Euclide produce due sequenze di numeri $q_1,q_2...$ e $r_1,r_2,...$ $\in \mathbb{Z}$ tali che 
        \begin{itemize}
          \item $a=bq+r$
          \item $b=q_1r+r_1$
          \item $r_i=q_{i+2}r_{i+1}+r_{i+2}$
        \end{itemize}
        Sia $r_k=0$ Sia $c\in \mathbb{P}$ tali che c|b e c|a. Ma allora c|r(=a-bq) $\Rightarrow$ c|$r_1$ (= b-$q_1$r)$\Rightarrow$ c|$r_i$ per
        $\forall i=1,...,k-2$ Pertanto $2_{k-1}$=MCD(a,b)
        E.g \\ 
        Calcolare il massimo comune divisore (153,126), Applichiamo l algoritmo:
        $153=1*126+27$ \\ 
        $126=4*27+18$ \\
        $27=1*18+9 \Rightarrow$ 9 e' il nostro massimo comune divisore \\
        $18=2*9+0$ \\
    \end{flushleft}
    \subsection{Conseguenze dell Algoritmo di Euclide}
    \begin{flushleft}
     Siano a,b,q,r come in A.E  \\ 
      \textbf{Osservazione:} Se $r>0 \Rightarrow$ MCD(a,b)=MCD(b,r)
    \end{flushleft}
    \subsubsection{Identita' di Bezout}
    \begin{flushleft}
      Siano $a,b \in \mathbb{P}.$ Allora $\exists x,y \in \mathbb{Z}$ tali che
      \begin{equation}
        (a,b)=ax+by
      \end{equation}
      Dimostrare sia $a\geq b$. Induzione completa su $b\geq 1$. \\ 
      \textbf{Passo base} \\ 
      Se b=1 $\Rightarrow$ b|a $\Rightarrow$ (a,b) = b $\Rightarrow$ a*0+b*1 $\Rightarrow$ ok \\
      Supponiamo la tesi vera per tutti i numeri $<b$.Allora $\exists q,r \in \mathbb{Z}$ tali che a=bq+r e $0\leq r < b$.
      \begin{itemize}
        \item Se $r=0$ $\Rightarrow$ b|a $\Rightarrow$ (a,b) = b $\Rightarrow$ ok
        \item Se $r>0  \Rightarrow (a,b)=(b,r).$ Ma $r<b \Rightarrow_{induzione} \exists x,y \in \mathbb{Z}$ tali che: $(b,r)=bx+2y$
      \end{itemize}
      Ma allora
      \begin{equation}
        (a,b)=bx+2y=bx+(a-b*q)y=b(x-qy)+ay. \square
      \end{equation}
    \end{flushleft}
    \subsubsection{Proprieta}
    \begin{flushleft}
      Siano $a,b \in \mathbb{P}$. Allora
      \begin{equation}
        (a,b)=min\{ax+by \in \mathbb{P}:x,y \in \mathbb{Z}\}
      \end{equation}
    \end{flushleft}
    \subsubsection{Proprieta}
    \begin{flushleft}
      Siano $a,b,p \in \mathbb{P}$ \\ 
      p primo tale che p|ab. Allora p|a o p|b.
      \textbf{Dimostrazione}: 
      \begin{itemize}
        \item Se p|a $\Rightarrow$ ok \\ 
        \item Se p not| a $\Rightarrow$ a,p=1 $\Rightarrow$ c|p $\Rightarrow$ c|a e c=1 o c=p
      \end{itemize}
      $\exists x,y \in \mathbb{Z}$ tali che 1=ax+py $\Rightarrow$ b=abx+pby $\Rightarrow$ p|b $\square$
      \textbf{Osservazioni}
      \begin{itemize}
        \item Falsa in generale se p non e' primo
        \item Analogamente si dimostra che se $m,a,b \in \mathbb{P}$ allora m|ab e (m,a)=1 $\Rightarrow$ m|b
      \end{itemize}
    \end{flushleft}
    \subsubsection{Teorema Fondamentale dell' Aritmetica}
    \begin{flushleft}
      Sia $n\in \mathbb{P}$. Allora n non puo' essere espresso in uno ed un solo modo come prodotto di numeri primi , a parte l'ordinamento dei fattori\\ 
      \textbf{Dimostrazione}:se $n \geq 2$. Sia n=2 e siano $p_1,p_2$ primi tali che $2=p_1*p_2 \Rightarrow p_1 \mid 2 \Rightarrow p_1 \leq 2 \Rightarrow p_2 = 2$
      \begin{equation}
       n=p_1*...*p_2=q_1*...*q_2 
      \end{equation}
      Allora $p_1 \mid m \Rightarrow p_1 \mid q_1*...*q_s \Rightarrow \exists 1\leq i < s : \quad p_1 \mid q_i \Rightarrow p_1=q_i$\\ 
      Ma allora
      \begin{equation}
        \frac{n}{p_1}=p_2*...*p_r=q_1*...*q_{i-1}*...q_s 
      \end{equation}
      ma $\frac{n}{p} < n \Rightarrow $ le due fattorizazioni coincidono a meno dell'ordine dei fattori, anche quelli in (15) coincidono a numeri
      dell'ordine dei fattori $\square$.
    \end{flushleft}
    \subsection{Equazione Diofane lineare}
    \subsubsection{Teorema}
    \begin{flushleft}
     Allora $\exists x,y \in \mathbb{Z}$ tali che
     \begin{equation}
       ax+by=n \quad (\square)
     \end{equation}
     se e solo se $(a,b) \mid h$\\ 
     \textbf{Dimostrazione}: Se $\exists x,y \in \mathbb{Z}$ tale che la formula vale, poiche' $(a,b)\mid a$ e $(a,b) \mid ab$ $\Rightarrow$ $(a,b)\mid h$\\ 
     Viceversa se $(a,b)\mid n \Rightarrow \exists k \in \mathbb{Z}: \quad n=k(a,b)$ Ma per 2.7.1 $\Rightarrow \exists x,y \in \mathbb{Z}$ tale che
     \begin{equation}
       ax+by=(a,b)
     \end{equation}
      Quindi axk + byk=n $\square$
    \end{flushleft}
    \subsubsection{Teorema}
    \begin{flushleft}
      Siano $a,b \in \mathbb{P}$. Allora le soluzioni $x,y \in \mathbb{Z}$ di ax+by=n sono tutte (se esistono, cioe' se $(a,b)\mid n$) della forma \\
      \begin{equation}
        \begin{cases}
          x= x_0 - \frac{b}{(a,b)}t \\
          y= y_0 - \frac{a}{(a,b)}t
        \end{cases}
      \end{equation}
      se hai una soluzione di una delle 2 equazioni puoi ricavare l altra.
      \\ 
      Doce $t \in \mathbb{Z} \quad x_0,y_0 \in \mathbb{Z}$ e' una soluzione di ($\square$).
    \end{flushleft}
     \subsection{Le classi di resto} 
     \begin{flushleft}
       Sia $n \in \mathbb{P}$. Poniamo una relazione $\equiv_n$ su $\mathbb{Z}$ in questo modo
       \begin{equation}
         a \equiv_n b \iff n\mid (b-a)
       \end{equation}
       $\forall a,b \in \mathbb{Z} \quad "\equiv_n"$ si dice "relazione di congruenza modulo n"( si scrive anche $a \equiv b(modn)$
     \end{flushleft}
     \subsubsection{Proprieta}
     \begin{flushleft}
       $\equiv_n$ e' di equivalenza. Dimostrazione: vedi 1.4 (n=3) $\square$\\ 
       Le classi di equivalenza rispetto $a \equiv_n$ si dicono classi di resto (modulo n), scritto $[a]_n$.E.g
       \begin{equation}
         \begin{split}
           [6]_8 & = \{6,14,-2,22,-10,...\}\\
           & = \{ b\in \mathbb{Z}:b \equiv_8 6 \}
         \end{split}
       \end{equation}
       Ci sono n classi di resto modulo n e cioe' $[0]_n,[1]_n...[n-1]_n$ \\ 
       Oss: Siano $a,b,c,d \in \mathbb{Z}$ e $n \in \mathbb{P}$ allora
       \begin{equation}
         a \equiv_n c \land b\equiv_n d \Rightarrow a+b \equiv_n c+d
       \end{equation}
       \begin{equation}
         a \equiv_n c \land b\equiv_n d \Rightarrow ab \equiv_n cd
       \end{equation}
       Ci suggerisce le seguenti definizioni.
       Poniamo 
       \begin{equation}
         [a]_n + [b]_n =^{def} [a+b]_n
       \end{equation}
       \begin{equation}
         [a]_n * [b]_n =^{def} [ab]_n
       \end{equation}
       (Somme e prodotto di classi di resto)\\ 
       Le classi di resto si comportano come numeri tranne quando:
       \begin{equation}
        [a]_n * [k]_n = [b]_n * [k]_n \land [k]_n \neq [0]_n
       \end{equation}
       questo non implica $[a]_n = [b]_n$ \\ 
       E.g
       \begin{equation}
         [6]_8 [4]_8 = [2]_8 [4]_8 \land [4]_8 \neq [0]_n
       \end{equation}
        ma $[2]_8 \neq [6]_8$
     \end{flushleft}
     \subsubsection{Proprieta}
     \begin{flushleft}
      Siano $a,b,k \in \mathbb{Z}$ e $n \in \mathbb{P}$ tali che (n,k)=1. Allora
       \begin{equation}
        [a]_n*[k]_n=[b]_n*[k]_n \Rightarrow [a]_n=[b]_n 
       \end{equation}
       Dimostrazione: Abbiamo che \\
       $[ak]_n=[bk]_n \Rightarrow a*k=b*k \Rightarrow n\mid k(b-a) \Rightarrow_{(n,k)=1}n\mid (b-a) \Rightarrow a \equiv_n b \Rightarrow [a]_n \equiv [b]_n \square$
     \end{flushleft}
     \begin{flushleft}
       Siano $a \in \mathbb{Z}$ e $n \in \mathbb{P}$ \\ 
       Definizione: un' inversa moltiplicativa di $[a]_n$ e' una $[b]_n$ tale che:
       \begin{equation}
         [a]_n*[b]_n=1
       \end{equation}
     \end{flushleft}
     \subsubsection{Proprieta}
     \begin{flushleft}
       Siano $a \in \mathbb{Z} \quad n \in \mathbb{P}$ tali che (a,n)=1. Allora $\exists ! [b]_n$ tale che
       \begin{equation}
         [a]_n*[b]_n=[1]_n
       \end{equation}
       \textbf{Dimostrazione:}Per Bezout $\Rightarrow \quad \exists x,y \in \mathbb{Z}: ax+by=1 \Rightarrow [a]_n*[x]_n = [ax]_n=[1-ny]_n=[1]_n$ \\ 
       Sia $[y]_n$ tale che $[a]_n*[y]_n=[1]_n$. Allora $[a]_n *[x]_n=[a]_n*[y]_n \to$ per 2.9.2 $\to [x]_n=[y]_n \square$
     \end{flushleft}
     \subsection{Le funzioni di Eulero}
     \begin{flushleft}
       \begin{equation}
         \Phi (n)=\mid \{ i\in \mathbb{P}: 1 \leq i \leq n, (n,i)=1\}\mid
       \end{equation}
       E.g
       \begin{equation}
         \Phi (8) =4.
       \end{equation}
       Osservazione $p\in \mathbb{P}$, p primo $\Rightarrow \Phi (p)=p-1$
     \end{flushleft}
     \subsubsection{Proprieta}
     \begin{flushleft}
      Siano p,q primi $p \neq q$. Allora 
       \begin{equation}
         \Phi (p*q)=(p-1)(q-1)
       \end{equation}
       \textbf{Dimostrazione:} Abbiamo che 
       \begin{equation}
         \Phi (p*q)=pq-\mid \{1 \leq i \leq pq : (pq,i) \geq 2\} \mid 
       \end{equation}
       Sia $r\in \mathbb{P}$, r primo allora:
       \begin{equation}
         r \mid i \land r \mid pq \iff r \mid i \land (r\mid p \lor r\mid q) \iff r\mid i \land (r=p \lor r=q)
       \end{equation}
       Quindi $p \mid i \lor q \mid i$ \\ 
       $i \in \{ p, 2p,3p,...(q-1)p,qp,q,2q,...(p-1)q\} \Rightarrow $ ci sono quindi p+q+1 tali i
     \end{flushleft}
     \subsubsection{Teorema}
     \begin{flushleft}
       Sia $n \in \mathbb{P}$ e sia $n = p^{a_1}_1*....*p^{a_r}_r$ la sua decomposizione in numeri primi:
       \begin{equation}
         \Phi (n) = n(1-\frac{1}{p_1})*...(1-\frac{1}{p_r})
       \end{equation}
        Dimostrazione vedi capitolo 3 $\square$ 
     \end{flushleft}
     \subsubsection{Con}
     \begin{flushleft}
       Siano $a,b \in \mathbb{P}$ tali che (a,b)=1. Allora:
       \begin{equation}
         \Phi (ab)=\Phi (a) * \Phi (b)
       \end{equation}
       Dimostrazione segue da 2.10.2 $\square$ \\ 
       Sia $n \in \mathbb{P}$. Poniamo
       \begin{equation}
         E_n = \{[i]_n: i\in \mathbb{Z} \quad (i,n)=1
       \end{equation}
      Notiamo che se $a,b \in \mathbb{Z}$ tali che $[a]_n=[b]_n$ allora:
       \begin{equation}
         (a,n)=1 \iff (b,n)=1
       \end{equation}
       Infatti, poiche' le classsi sono uguali $[a]_n=[b]_n \Rightarrow n \mid (b-a) \Rightarrow \exists k \in \mathbb{Z}$ tali che a=b+kn. Sia
       $r \in \mathbb{P}$, r primo. Allora
       \begin{equation}
        r\mid a \land r\mid n \iff r\mid b \land r\mid n
       \end{equation}
       Osservazione: $\mid E_n \mid = \Phi(n)$
     \end{flushleft}
     \subsubsection{Proprieta}
     \begin{flushleft}
       Siano $n \in \mathbb{P} \quad k \in \mathbb{Z}$ tali che (k,n)=1. Allora la funzione $f:E_n \to E_n$ definita da $f([i]_n)=[i]_n*[k]_n$ \\ 
       $\forall [i]_n \in E_n$ e' biunivoca
       Dimostrazione: Sia $[i]_n \in E_n \Rightarrow (i,n)=1 \Rightarrow [i*k]\in E_n \Rightarrow [i]_n*[k]_n \in E_n$ \\ 
       Siano $[i]_n[j]_n \in E_n $ tali che $ [i]_n*[k]_n=[j]_n*[k]_n \Rightarrow [i]_n = [j]_n$ quindi e inniettiva \\ 
       Sia $[a] \in E_n$ Poiche' (k,n)=1 $\Rightarrow \exists [b]_n \in E_n$ tali che \\ 
       $[k]_n*[b]_n =[1]_n$ Allora $[ab]_n \in E_n \Rightarrow [ab]_n*[k]_n=[a]_n[b]_n[k]_n= [a]_n*[1]_n=[a]_n \square $ e' surriettiva
     \end{flushleft}
     \subsubsection{Teorema di Eulero}
     \begin{flushleft}
       Siano $k \in \mathbb{Z}$ e $n \in \mathbb{P}$, tali che (n,k)=1 allora:
       \begin{equation}
         k^{\Phi(n)} \equiv_n 1 
       \end{equation}
     \end{flushleft}
     \begin{flushleft}
       Dimostrazione: Sappiamo da 2.10.4 che la funzione 
       \begin{equation}
         [a]_n \to [a]_n*[k]_n
       \end{equation}
       e' una biezione di $E_n$. Sia
     \end{flushleft}
     \begin{equation}
       E_n = \{ [k]_n,...,[k_r]_n\} \Rightarrow r=\Phi (n)
     \end{equation}
     \begin{flushleft}
      Dimostrazione sappiamo da 2.10.4 che ha funzione
       \begin{equation}
         [a]_n \to [a]_n*[k]_n
       \end{equation}
       e' una biezione di $E_n$. Sia
       \begin{equation}
         E_n = \{ [k]_n,....,[k_r]_n\} \Rightarrow r = \Phi (n)
       \end{equation}
       Ma allora:
       \begin{equation}
         E_n = \{ [k_1 * k]_n,....,[k_r * k_]n\} 
       \end{equation}
       Quindi:
       \begin{equation}
         [k_1]_n*...*[k_r]_n=[k_1]_n*...*[k_r]_n*[k]_n
       \end{equation}
       \begin{equation}
         [1]_n*[k_1]_n*...*[k_r]_n=[k_1]_n*...*[k_r]_n*[k]^r_n = [1]_n=[k^r]_n. \square
       \end{equation}
     \end{flushleft}
     \subsubsection{Teorema di Fermat-Eulero}
     \begin{flushleft}
       Siano $k \in \mathbb{Z}$ e $p \in \mathbb{P}$, primo, tale che $p \nmid k$ allora:
       \begin{equation}
         k^{p-1} \equiv_n 1
       \end{equation}
       Dimostrazione: Basta porre n=p in 2.10.5 $\square$
     \end{flushleft}
     \subsection{Il codice RSA}
     \begin{flushleft}
      Problema fondamentale della crittografia: \\ 
       Spedire un mesaggio da  A a B in modo che solo B possa leggerlo (decifrarlo).
     \end{flushleft}
     \begin{itemize}
       \item \textbf{Preparazione:} B sceglie 2 primi p e q, p $\neq$ q e calcola:
         \begin{equation}
          n=pq
         \end{equation}
         Quindi sceglie $e \in \mathbb{P}$ tali che 
         \begin{equation}
           (e,(p-1)(q-1))=1
         \end{equation}
         Infine calcola l'inversa moltiplicativa $[d]_{(p-1)(q-1)}$ di $[e]_{(p-1)(q-1)}$\\ 
         B pubblica n,e  e tiene segreti p,q,d
       \item \textbf{Codifica:} A prende un mesaggio $1 \leq m \leq n,(m,n)=1$ e calcola
         \begin{equation}
           [m^{\sim}]_n=[m^e]_n
         \end{equation}
         e spedisce $m^{\sim}$
       \item \textbf{Decodifica:} B riceve $m^{\sim}$ e decodifica calcolando
         \begin{equation}
           [m^{\sim d}]_n
         \end{equation}
     \end{itemize}
     \begin{flushleft}
       Osservazione: A e B non si scambiano niente. Perche pensiamo che sia difficile rompere RSA? Per rompere RSA dovremmo:
     \end{flushleft}
     \begin{itemize}
       \item Fattorizzare n ( caso impossibile se n e' grande)
       \item risolvere l equazione della forma:
         \begin{equation}
           [x^e]_n=[m^{\sim}]_n
         \end{equation}
     \end{itemize}
     \begin{flushleft}
      Se e=2 $\Rightarrow$ teoria della recipocita' quadratica \\ 
       Se e $\geq$ 3 $\Rightarrow$ ricerche attuali
     \end{flushleft}
     \begin{flushleft}
      Attualmente grande va a significare piu di 10000 cifre
     \end{flushleft}
     \subsubsection{Altri tipi di codici}
     \begin{itemize}
       \item Codice romano (Debolezza $\to$ analisi delle frequenza)
       \item Codice di Turing (Debolezza $\to$ $MCD(mp,m_1p_1,...)=p$
     \end{itemize}
     RECUPERARE LE LEZIONI
     \section{Combinatorie Enumerative}
     \subsection{Problema fondamentale della combinatoria enumerativa}
     \begin{flushleft}
       Data una sequenza di insiemi $\{A_n\}_{n \in \mathbb{N}}$, calcolare $\mid \{A_n\}_{n \in \mathbb{N}}\mid  $,
     \end{flushleft}
      Cosa vuol dire "calcolare"?
     \begin{enumerate}
      \item Una ricorsione (e.g $\mid A_n\mid = \mid A_{n-1} \mid + \mid A_{n-2} \mid$ se $n \geq 2$)
      \item Una formula (e.g $\mid A_n \mid = 2^n$)
      \item Una funzione generatrice (cioe' una funzione $f: \mathbb{R} \to \mathbb{R}$,$C^\infty$ in x=0, tale che lo sviluppo in serie di Taylor di f(x) in x=0 e'
        \begin{equation}
          \sum_{n\geq 0} \mid A_n \mid  x^n
        \end{equation}
      \end{enumerate}
      \subsection{Proprieta' Fondamentale}
      \begin{flushleft}
        Osservazione: A,B insiemi, $f:A\to B$ biunivoca
        \begin{equation}
          \mid A \mid = \mid B \mid 
        \end{equation}
      \end{flushleft}
      \begin{flushleft}
        Def: La potenza di A elevato alla B e'
        \begin{equation}
          A^B=\{f:B\to A\}
        \end{equation}
      \end{flushleft}
      \subsubsection{Proprieta'}
      A,B insiemi allora:
      \begin{enumerate}
        \item $\mid AxB \mid = \mid A \mid * \mid B \mid$
        \item $\mid A^B \mid = \mid A \mid^{\mid B \mid}$
        \item $\mid A \cup B \mid = \mid A \mid + \mid B \mid - \mid A \cap B \mid$
      \end{enumerate}
      Dimostrazione chiara $\square$
      \subsection{Coefficienti Binomiali}
      \begin{flushleft}
        Sia $n\in \mathbb{P}$. Ricordiamo che: $[n]=\{1,2,3,4...n\}$
      \end{flushleft}
      \subsubsection{Proprieta'}
      \begin{flushleft}
        Sia $n \in \mathbb{P}$ Allora:
        \begin{equation}
          \mid P([n]) \mid = 2^n
        \end{equation}
      \end{flushleft}
      \begin{flushleft}
        Dimostrazione: Costruiamo una funzione
        \begin{equation}
          \phi : P([n]) \to [2]x[2]x...x[2]
        \end{equation}
        Ponendo
        \begin{equation}
          \phi (A) = \epsilon_1,...,\epsilon_n
        \end{equation}
        Dove $\epsilon_i$ = 
        \begin{itemize}
          \item 1, se $i \notin A$
          \item 2, se $i \in A$
        \end{itemize}
      $\forall i \in [n]$ e $\forall A \subseteq [n]$ (e.g $\phi(\{1,3,4\})=\{2,1,2,2,1\}$
      \end{flushleft}
      \begin{flushleft}
        Allora $\phi$ e' biunivoca. Quindi
        \begin{equation}
          \mid P(n) \mid = \mid [2]x...x[2] \mid = \mid [2] \mid * \mid [2] \mid * ... * \mid [2]^n \mid = 2^n. \square
        \end{equation}
      \end{flushleft}
      \begin{flushleft}
        Sia $n \in \mathbb{Z}$. Def il coefficiente binomiale (di grado n) se $n \geq 1$ e':
        \begin{equation}
          \begin{pmatrix}
            x \\ 
            n
          \end{pmatrix} = \frac{x(x-1)(x-2)...(x-n+1)}{n!}
        \end{equation}
        $\begin{pmatrix}
          x \\ 
          0
        \end{pmatrix}$=1,
        $\begin{pmatrix}
          x \\ 
          n
        \end{pmatrix}$=0,se $n \leq 0$
      \end{flushleft}
      \begin{flushleft}
        (Letto "x binomiale n" o "x sceglie n")
      \end{flushleft}
      \begin{flushleft}
        Oss $\begin{pmatrix}
          x \\ 
          n
        \end{pmatrix} \in \mathbb{Q}[x]$
      \end{flushleft}
      \subsubsection{Proprieta'}
      \begin{flushleft}
        Siano $n,k \in \mathbb{N} \quad 0 \leq k \leq n$. Allora
        \begin{equation}
          \mid \{ A \subseteq [n]: \mid A \mid = k\} \mid = \begin{pmatrix}
            n \\ 
            k
          \end{pmatrix}
        \end{equation}
      \end{flushleft}
      \begin{flushleft}
        Dimostrazione su carta
      \end{flushleft}
      \subsubsection{Proprieta'}
      \begin{flushleft}
        Se $n \in \mathbb{P}$ Allora
      \end{flushleft}
      \begin{equation}
        (1+x)^n=\sum_{k=0}^n \begin{pmatrix}
          n \\
          k
        \end{pmatrix} x^k
      \end{equation}
      \begin{flushleft}
        Dimostrazione per induzione $n \geq 1$ se n=1 ok, Sia $n \geq 2$
      \end{flushleft}
        \begin{equation}
          \begin{split}
            & (1+x)^n=(1+x)(1+x)^{n-1}= \\
             = & (1+x)^n(\sum_{k=0}^n \begin{pmatrix}
              n \\
              k
             \end{pmatrix} x^k)=
             \sum_{k=0}^{n-1} \begin{pmatrix}
              n-1 \\
              k
             \end{pmatrix} x^k+
             \sum_{k=0}^{n-1} \begin{pmatrix}
              n-1 \\
              k
             \end{pmatrix} x^{k+1}=\\
             = & 
             \sum_{k=0}^{n} \begin{pmatrix}
              n-1 \\
              k
             \end{pmatrix} x^k+
             \sum_{k=1}^{n} \begin{pmatrix}
              n-1 \\
              k-1
             \end{pmatrix} x^{k}=\\
             = & \sum_{k=0}^n[\begin{pmatrix}
              n-1 \\ 
               k
             \end{pmatrix}x^k+\begin{pmatrix}
              n-1 \\ 
               k-1
             \end{pmatrix}] x^k= \sum_{k=0}^n \begin{pmatrix}
               n \\ 
               k
             \end{pmatrix}x^k \square
          \end{split}
        \end{equation}
        \subsubsection{Proprieta'}
        \begin{flushleft}
          Sia $n \in \mathbb{P}$ Allora:
        \end{flushleft}
        \begin{equation}
          \mid \{ A \subseteq [n]: \mid A \mid \textrm{e' pari} \} \mid = \mid \{ A \subseteq [n]: \mid A \mid \textrm{e' dispari} \} \mid 
        \end{equation}
        \subsection{Il principio di Inclusione-Esclusione}
        \begin{flushleft}
          Ricordiao che (3.2)
        \end{flushleft}
        \begin{equation}
          \mid A \cup B \mid = \mid A \mid + \mid B \mid - \mid A \cap B \mid
        \end{equation}
        \begin{flushleft}
          A e B sono insiemi finiti
        \end{flushleft}
        \begin{flushleft}
          Siano A,B,C insiemi finiti allora:
        \end{flushleft}
        \begin{equation}
          \mid A \cup B \cup C \mid = \mid A \mid + \mid B \mid + \mid C \mid - \mid A \cap B \mid -\mid A \cap C \mid -\mid B \cap C \mid + \mid A \cap B \cap C \mid
        \end{equation}
        \begin{flushleft}
          Allo stesso modo si ottiene. Siano $A_1, A_2,..., A_n$ insiemi finiti 
        \end{flushleft}
        \begin{flushleft}
          Dato $T \subseteq [n], T=\{ t_1,...,t_r \}$ poniamo 
        \end{flushleft}
        \begin{equation}
          A_T= A_{t_{1}} \cap A_{t_{1}} \cap A_{t_{2}} \cap A_{t_{3}} \cap...\cap A_{t_{r}}
        \end{equation}
        \subsubsection{Teorema}
        \begin{flushleft}
          Siano $n \in \mathbb{P}$ e $A_1,...,A_n$ insiemi finiti. Allora
        \end{flushleft}
        \begin{equation}
          \mid A_1 \cap ... \cap A_n \mid = \sum_{T \subseteq [n]} (-1)^{\mid T \mid -1}*\mid A_t \mid 
        \end{equation}
        \begin{flushleft}
          Sia $n \in \mathbb{P}$ e sia
        \end{flushleft}
        \begin{equation}
          n = p^{a_{1}}*...*p^{a_{r}}
        \end{equation}
        \begin{flushleft}
          La fattorizzazione di n in numeri primi allora:
        \end{flushleft}
        \begin{equation}
          \Phi (n) = n(1-\frac{1}{p_1})...(1-\frac{1}{p_r})
        \end{equation}
        \begin{flushleft}
          Dimostrazione: Notiamo che 
        \end{flushleft}
        \begin{equation}
          \Phi (n) = n-\mid \{ 1 \leq i \leq n:(n,i) \geq r \} \mid
        \end{equation}
        \begin{flushleft}
          Poniamo che
        \end{flushleft}
        \begin{equation}
          \begin{split}
            & A_1=\{ 1 \leq i \leq n : p_i \mid i \} \\
            &...\\ 
            &... \\ 
            &... \\
            & A_r=\{ 1 \leq i \leq r : p_r \mid i \}
          \end{split}
        \end{equation}
        \begin{flushleft}
          Ma allora
        \end{flushleft}
        \begin{equation}
          \{ 1 \leq i \leq n: (n,i) \geq 2 \}= A_1 \cup ... \cup A_r
        \end{equation}
        \begin{flushleft}
          Se $i \in A_r \Rightarrow p_r \mid i \land p_r \mid n \Rightarrow (n,i) \geq 2$\\ 
          Viceversa $(n,i) \geq 2 \Rightarrow \exists q$ tale che $q\mid n$ e $q \mid i \Rightarrow q=p_r$ \\ 
          Quindi per 3.4.1
        \end{flushleft}
        \begin{equation}
          \mid A_1 \cup ... \cup A_r \mid = \sum_{T \subseteq [r]} (-1)^{\mid T\mid -1}*\mid A_r \mid
        \end{equation}
        Allora
        \begin{equation}
          \begin{split}
            A_T  & = A_t \cap ... \cap A_{t_{k}} = \\ 
            & = \{ 1 \leq i \leq n: p_{t_{1}} \mid i....p_{t_{k}} \mid i \} = \\ 
            & = \{ 1 \leq i \leq n: (p_{t_{1}}*...*(p_{t_{k}}\mid i \} = \\ 
            & = \{ (p_{t_{1}}*...*p_{t_{k}}),2(p_{t_{1}}*...*p_{t_{k}}),...,\frac{n}{(p_{t_{1}}*...*p_{t_{k}}})(p_{t_{1}}*...*p_{t_{k}} )\} = \\
            & = \mid A_T \mid = \frac{n}{(p_{t_{1}}*...*(p_{t_{k}}}
          \end{split}
        \end{equation}
        Pertanto
        \begin{equation}
          \begin{split}
            & \mid A_1 \cup ... \cup A_r \mid = \sum_{T \subseteq [r]} (-1)^{\mid T \mid -1}* \frac{n}{(p_{t_{1}}*...*(p_{t_{k}}} = \\ 
            &  =-n(\sum_{T \subseteq [r]} (-1)^{\mid T \mid -1}* \frac{1}{(p_{t_{1}}*...*(p_{t_{k}}}-1) = \\ 
            & = -n((1-\frac{1}{p_1})(1-\frac{1}{p_2})...(1-\frac{1}{p_r})-1) \square
          \end{split}
        \end{equation}
        \subsection{Composizioni}
        \begin{flushleft}
          Siamo $n,k \in \mathbb{P}$
        \end{flushleft}
        \begin{flushleft}
          Def: Una composizione di n in k parti e' una sequenza $(a,...,a_k = \mathbb{P}x...x\mathbb{P})$ tali che $a+...+a_r=n$
        \end{flushleft}
        \begin{flushleft}
         E.g. La composizione di 5 in 3 parti. Siano:
          \begin{equation}
            (3,1,1),(1,3,1),(1,1,3),(2,2,1),(2,1,2),(1,2,2)
          \end{equation}
        \end{flushleft}
        \subsubsection{Proprieta'}
        \begin{flushleft}
          Siano $n,k \in \mathbb{P}$ Allora ci sono
          \begin{equation}
            \begin{pmatrix}
              n-1 \\ 
              k-1
            \end{pmatrix}
          \end{equation}
          Dimostrazione: C'e' una biezione tra sottoinsiemi di [n-1] di cardinalita k-1 e composzioni di n in k parti
        \end{flushleft}
        \begin{flushleft}
          $...\mid ... \mid ...\mid ...... \mid $(n pallini)
        \end{flushleft}
        n-1 =  spazi tra un pallino e l altro, k-1 = barre $\square$
        \begin{flushleft}
          Def: Una composizione debole di n in k parti e' una sequenza $(a_1,...,a_k) \in \mathbb{N}^+$ tale che $a_1+...+a_k=n$ (adesso che anche lo zero)
        \end{flushleft}
        \subsubsection{Proprieta}
        \begin{flushleft}
          Siano $n,k \in \mathbb{P}$ Allora ci sono 
          \begin{equation}
            \begin{pmatrix}
              n+k-1 \\ 
              k-1
            \end{pmatrix}
          \end{equation}
          composizioni deboli di n in k parti
        \end{flushleft}
        \begin{flushleft}
          Dimostrazione: C'e' una biezione tra composizioni deboli di n in k parti e composizioni di n+k in k parti.Infatti 
          \begin{equation}
            \begin{split}
              & (a,...,a_k) \\
              & \mid \\ 
              & (a+1,...,a_k+1)
            \end{split}
          \end{equation}
        \end{flushleft}
        \subsection{Coefficienti multinomiali}
        \begin{flushleft}
          Siano $n,k \in \mathbb{P}$ e sia $(a,...,a_k)$ una composizione di n in k parti
        \end{flushleft}
        \begin{flushleft}
          Def: Coefficiente multinomiale si scrive 
          \begin{equation}
            \begin{pmatrix}
              n \\ 
              a_1,...,a_k
            \end{pmatrix}
          \end{equation}
          di un rispetto a $(a_1,...,a_k)$ e' il numero di modi di assegnare ogni $ i \in [n]$ (palline) ad una di k categorie, $C_1,...,C_k$ 
          in modo che esattamente $a_j$ numeri vengono assegnati alla cateogoria j ( $\forall j, j=1....k$) $\to$ scatole numerate
        \end{flushleft}
        \begin{flushleft}
          Osservazione: \begin{equation}
           \begin{pmatrix}
            n \\ 
             a,a_2
           \end{pmatrix}=\begin{pmatrix}
            n \\ 
             a
           \end{pmatrix}
          \end{equation}
        \end{flushleft}
        \subsubsection{Proprieta'}
        \begin{flushleft}
          Siano $n \in \mathbb{P}$ e $(a_1,...,a_k)$ composizione di n in k parti. Allora:
          \begin{equation}
            \begin{pmatrix}
              n \\ 
              a_1,...,a_k
            \end{pmatrix}=\frac{n!}{a_1!,...,a_k!}
          \end{equation}
        \end{flushleft}
        \begin{flushleft}
          \textbf{Dimostrazione} Possiamo scegliere numeri da mattere in $C_1$ in $\begin{pmatrix}
            n \\ 
            a_1
          \end{pmatrix}$ nodi. Rimangono n-a numeri. Possiamo scegliere i numeri da mettere in $C_2$ $\begin{pmatrix}
            n-a_1 \\ 
            a_2
          \end{pmatrix}$ etc ...
        \end{flushleft}
        \begin{flushleft}
          Infine rimangono $n-a_1-...-a_k$ numeri possiamo scegliere i numeri da mettere in $C_k$ in $\begin{pmatrix}
            n-a_1-...-a_k \\ 
            a_k
          \end{pmatrix}$ modi. In totale avremmo:
          \begin{equation}
            \begin{pmatrix}
              n \\ 
              a_1,...,a_k
            \end{pmatrix}= 
            \begin{pmatrix}
              n \\ 
              a_1
            \end{pmatrix}
            \begin{pmatrix}
              n-a_1 \\ 
              a_2
            \end{pmatrix}*...*
            \begin{pmatrix}
              n-a_1-...a_k \\ 
              a_k
            \end{pmatrix} \square
          \end{equation}
        \end{flushleft}
        \subsection{Sequenze con ripetizioni}
        \begin{flushleft}
          Siano $n \in \mathbb{P}$. Sappiamo che di [n] di cardinalita' $k \iff$ stringa binaria di lunghezza n con k "1" 
        \end{flushleft}
          E.g: [n]=5 k=3
        \begin{flushleft}
          $\{2,4,5\}="01011"$
        \end{flushleft}
        \begin{flushleft}
          E' naturale chiedersi sia $(a_1,...,a_k)$ una composizione
        \end{flushleft}
        \begin{flushleft}
          Quante stringhe k-arie di lunghezza n ci sono che hanno $a_1 1,a_2 2, a_3 3,a_k k$
        \end{flushleft}
        E.g n=4 k=3
        \begin{flushleft}
          $(a_1,a_2,a_3)=(1,1,2)\to 12$ stringhe
        \end{flushleft}
        \begin{flushleft}
          1 viene ripetuto una volta \\
          2 viene ripetuto una volta \\ 
          3 viene ripetuto una volta  \\
        \end{flushleft}
        \begin{flushleft}
          Contare le possibili combinazioni che si possono ottenere
        \end{flushleft}
        \subsubsection{Proprieta'}
        \begin{flushleft}
          Siano $n \in \mathbb{P}, k \in \mathbb{N}$ e $(a_1,...,a_k)$ una composizione allora:
        \end{flushleft}
        \begin{equation}
          \begin{pmatrix}
            n \\ 
            a_1,...,a_k
          \end{pmatrix}
        \end{equation}
        \begin{flushleft}
          strighe k-arie di lunghezza n con $a_1 "1",a_2 "2",...,a_k "k"$
        \end{flushleft}
        
        \begin{flushleft}
          \textbf{Dimostrazione:} ce una biezione tra queste stringhe ed i modi di assegnare ogni $i \in [n]$ ad una di k categoria,
          in modo che $a_j$ numeri sono assegnati a categoria $C_j$ ($\forall j =1,...,k)$ \\ 
          Cioe' $x_1,x_2,x_3,...,x_n$ \\ 
          Assegnamo $i \in [n]$ alla categoria $C_j$ se e solo se $x_i=j$ $\forall j \in [k]$ E' una biezione
        \end{flushleft}
        \subsubsection{Proprieta'}
        \begin{flushleft}
          Sia $n \in \mathbb{P}$ allora 
        \end{flushleft}
        \begin{equation}
          \frac{1}{(1+x)^n}=\sum_{k \geq 0} \begin{pmatrix}
            -n \\ 
            k
          \end{pmatrix}x^k
        \end{equation}
        \begin{flushleft}
          Dimostrazione: Sia $k \geq 0$ allora:
        \end{flushleft}
        \begin{equation}
          (\frac{d}{dx})^k ((1+x)^{-n} \mid_{x=0} = (-n)(-n-1)(-n-2)...(-n-k-1) \to \sum_{k \geq 0} \frac{(\frac{d}{dx})^k((1+x)^{-n}\mid_{x=0}}{k!}x^k=
        \end{equation}
        \begin{equation}
          \sum_{k \geq 0} \begin{pmatrix}
            -n \\ 
            k
          \end{pmatrix} x^k \square
        \end{equation}
        \subsection{Something}
        \subsection{Ricorsioni Lineari a coefficienti costanti}
        \subsubsection{Teorema fondamentale dell'algebra}
        \begin{flushleft}
          Siano $n d \in \mathbb{P}$ e $a_0,...,a_d \in \mathbb{R},a_d \neq 0$. 
        \end{flushleft}
        \begin{flushleft}
          Allora $\exists a_1,...,a_d \in \mathbb{C} (r \in \mathbb{P}$ e $\exists d_1,...,d_r \in \mathbb{P}$ tali che
        \end{flushleft}
        \begin{equation}
          a_0+a_1*x+...+a_d*x^d=a_d(x-d_1)^{d_1}...(x-d_r)^{d_r} \iff d_1+...+d_r =d
        \end{equation}
        \begin{flushleft}
          Def: $\alpha_1,...,\alpha_r$ si dicono radici di $a_0+a_1*x+...+a_d*x^d$, e $d_1,...,d_r$ si dicono le moltiplcita' di $\alpha_1,...,\alpha_r$,
          rispettivamente.
        \end{flushleft}
        \subsubsection{Proprieta'}
        \begin{flushleft}
          Sia $P(x) \in \mathbb{R}[x]$ e sia $\alpha \in \mathbb{C}$, allora:
        \end{flushleft}
        \begin{equation}
          P(\alpha)=0 \iff (x-\alpha) \mid P(x)
        \end{equation}
        \begin{flushleft}
          Dimostrazione omessa $\square$
        \end{flushleft}
        \begin{flushleft}
          Se $A(x),B(x) \in \mathbb{R}[x]$ Allora si dice che A(x) divide B(x), scritto A(x) $\mid$ B(x) se $\exists C(x) \in \mathbb{R}[x]$ tale che
        \end{flushleft}
        \begin{equation}
          B(x)=A(x)*C(x)
        \end{equation}
        \begin{flushleft}
          Sia $f: \mathbb{N} \to \mathbb{R}$
        \end{flushleft}
        \begin{flushleft}
          Def: Si dice che f soddisfa una ricorsione lineare a coefficienti costanti. Se $\exists a_0,...,a_{d-1} \in \mathbb{R}$ $(d \in mathbb{P})$ tali che 
        \end{flushleft}
        \begin{equation}
          f(n+d)= a_{d-1}*f(n+d-1)+...+a_1*f(n+1)+a_0*f(n) \quad (*)
        \end{equation}
        \begin{flushleft}
          \textbf{Eruestia (e idea)}:
        \end{flushleft}
        \begin{flushleft}
          Calcolando i primi termini della sequenza $\{ f(n)\}_{n\in \mathbb{N}}$ si vede che f(n) sembra crescere esponenzialmente. \\ 
          Sia allora $\lambda \in \mathbb{C}$ tale che $f(n)=\lambda^n$ per $\forall n \in \mathbb{N}$. \\
          Una tale f(n) e' soluzione di $(*)$ se e solo se
        \end{flushleft}
        \begin{equation}
          \lambda^{n+d}=a_{d-1}*\lambda^{n+d-1}+...+a_1*\lambda^{n+1}+a_0*\lambda^n
        \end{equation}
        \begin{flushleft}
          Per $\forall n \in \mathbb{N}$, cioe' se e solo se:
        \end{flushleft}
        \begin{equation}
          \lambda^d=a_{d-1}*\lambda^{d-1}+...+a_1*\lambda+a_0
        \end{equation}
        \begin{flushleft}
          Cioe' se e solo se $\lambda$ e' radice di
        \end{flushleft}
        \begin{equation}
          x^d-a_{d-1}*x^{d-1}-...-a_1*x-a_0=0\quad (**)
        \end{equation}
        \begin{flushleft}
          Def $(**)$ si dice l'equazione caratteristica di $(*)$ ( o polinomio caratteristico)
        \end{flushleft}
        \subsubsection{Teorema}
        \begin{flushleft}
          Siano $f: \mathbb{N} \to \mathbb{R}$ e $d \in \mathbb{P}$ e $a_0,...,a_{d-1} \in \mathbb{R}$
        \end{flushleft}
        \begin{flushleft}
          Allora f soddisfa (ricorsione lineare a coefficenti costanti
        \end{flushleft}
        \begin{equation}
          f(n+d)=a_{d-1}*f(n+d-1)+...+a_1*f(n+1)+a_0*f(n)
        \end{equation}
        \begin{flushleft}
          $\forall n \in \mathbb{N}$ se e solo se $\exists$ polinomi $P_1(x)...P_n(x)$ tali che Deg$(P_i(x)) \leq d_1 -1 \quad (\forall i = i,...,r)$ e
        \end{flushleft}
        \begin{equation}
          f(n)= \sum^r_{i=1} P_i(n)(\gamma_i)^n
        \end{equation}
        \begin{flushleft}
          $\forall n \in \mathbb{N}$ dove $\gamma_1,...,\gamma_r \in \mathbb{C}$ sono le radici dell'equazione caratterstica $(**)$ e $d_1,...,d_r \in \mathbb{P}$
          sono le loro molteplicita'
        \end{flushleft}
        \begin{flushleft}
          Dimostrasione omessa $\square$
        \end{flushleft}
        \section{Somme e approsimazioni}
        \subsection{Annuita'}
        \begin{flushleft}
          Abbiamo vinto il super enalotto e abbiamo 2 scelte
          \begin{itemize}
            \item 1 milione di euro subito
            \item 50k euro/anno per 30 anni
          \end{itemize}
          Vediamo quale ci conviene
        \end{flushleft}
        \begin{flushleft}
          Sia x +- 50000 e sia p=inflazione. Allora in 30 anni riceveremo
        \end{flushleft}
        \begin{equation}
          x+x(1-p)+x(1-p)^2+...+(1-p)^{29}=x\sum_{i=0}^{29}(1-p)^i=x \frac{1+(1-p)^{30}}{1+(1-p)}=20x \quad (circa)
        \end{equation}
        \begin{flushleft}
          E se infiniti anni? Abbiamo
        \end{flushleft}
        \begin{equation}
          x \sum_{i \geq 0} (1-p)^i =^{3.7.3} x \frac{1}{1-(1-p)}=x \frac{1}{0,03}=1.666.661 \quad (circa)
        \end{equation}
        \begin{flushleft}
          Abbiamo usato il seguente
        \end{flushleft}
        \subsubsection{Lemma (Somma geometrica)}
        \begin{flushleft}
          Sia $n \in \mathbb{P}$ allora
        \end{flushleft}
        \begin{equation}
          \sum_{i=0}^n x^i = \frac{1-x^{n+1}}{1-x}
        \end{equation}
        \begin{flushleft}
          Dimostrazione: Si dimostra facilmente per induzione $\square$
        \end{flushleft}
        \begin{flushleft}
          Come indoviniamo una tale formula?
        \end{flushleft}
        \begin{flushleft}
          \textbf{Metodo della Peturbazione} \\ 
          Sia $S=1+x+x^2+...+x^n \Rightarrow x*S=x+x^2+...+x^{n+1}=x^{n+1}+(S-1)$
        \end{flushleft}
        \subsection{Somme polinomiali}
        \begin{flushleft}
          Come posso indovinare
        \end{flushleft}
        \begin{equation}
          \sum^n_{i=1}i=\begin{pmatrix}
            n+1 \\ 
            2
          \end{pmatrix}?
        \end{equation}
        \begin{equation}
          2\sum^n_{i=1}i=1+2+...+(n-1)+n+n+(n-1)+...+2+1=n(n+1)
        \end{equation}
        \subsubsection{Teorema}
        \begin{flushleft}
          Sia $f(x) \in \mathbb{R}[x]$ e sia $d=deg(f)$, allora $\exists g(x)\in \mathbb{R}[x]$ tale che
        \end{flushleft}
        \begin{equation}
          \sum_{i=0}^n f(i)=g(n)
        \end{equation}
        \begin{flushleft}
          $\forall n \in \mathbb{N}$ e $deg(g) \leq d+1$
        \end{flushleft}
        \begin{flushleft}
          Dimostrazione Omessa $\square$
        \end{flushleft}
        \begin{flushleft}
          Def g(n) si dice una formula chiusa per
        \end{flushleft}
        \begin{equation}
          \sum_{i=0}^n f(i)
        \end{equation}
        \subsection{Somme non polinomiali}
        \begin{flushleft}
          Come stimare:
        \end{flushleft}
        \begin{equation*}
          \sum_{i=1}^n \sqrt{i} ?
        \end{equation*}
        \subsubsection{Teorema}
        \begin{flushleft}
          Sia $f:\mathbb{R}^+ \to \mathbb{R}^+$, continua e monotona. Allora:
        \end{flushleft}
        \begin{equation*}
          f(1)+\int_1^n f(x)dx \leq \sum_{i=1}^n f(i) \leq f(n)+\int^n_i f(x)dx
        \end{equation*}
        \begin{flushleft}
          Se f e' crescente. Invece se f e' decrescente rovesciare le disuguaglianze
        \end{flushleft}
        \begin{flushleft}
          Dimostrazione: \\ 
          Sia f crescente allora
        \end{flushleft}
        \begin{equation*}
          \int_1^n f(x)dx=\sum_{i=1}^{n-1} \int^{i+1}_1 f(x)dx \leq \sum_{i=1}^{n-1} f(i+1)
        \end{equation*} 
        \begin{flushleft}
          Similmente
        \end{flushleft}
        \begin{equation*}
          \int_1^n f(x)dx=\sum_{i=1}^{n-1} \int^{i+1}_1 f(x)dx \geq \sum_{i=1}^{n-1} f(i)=\sum_{i=1}^n f(i) - f(n)
        \end{equation*}
        \begin{flushleft}
          Quindi
        \end{flushleft}
        \begin{equation*}
          f(1)+\int_1^n f(x)dx \leq \sum_{i=1}^n f(i) \leq f(n)+\int^n_i f(x)dx
        \end{equation*}
        \begin{flushleft}
          Si ottiene spostando i membri dell'equazioni sopra
        \end{flushleft}
        \begin{flushleft}
          \textbf{E.g} stimare 
          \begin{equation*}
            \sum_{i=1}^n \sqrt{i}
          \end{equation*}
          La funzione $\sqrt{x}$ e' continua e monotona crescente per $x>0$. Quindi per 4.3.1
        \end{flushleft}
        \begin{equation*}
          \sqrt{1}+\int^n_1 \sqrt{x}dx \leq \sum^n_{i=1} \sqrt{i} \leq \sqrt{n} + \int^n_1 \sqrt{x}dx
        \end{equation*}
        \begin{flushleft}
          Sviluppando gli integrali e sostituendo i valori otteniamo:
        \end{flushleft}
        \begin{equation*}
          1+(*)\frac{2}{3}n^{\frac{3}{2}}-\frac{2}{3} \leq \sum^n_{i=1} \sqrt{i} \leq n^{\frac{1}{2}}+(*)\frac{2}{3}n^{\frac{3}{2}}-\frac{2}{3} \quad \forall n \in \mathbb{N}
        \end{equation*}
        \begin{flushleft}
          Termini che vanno all'infinito piu' velocemente sono quelli con l asterisco (*).
          Noi adesso vogliamo calcolare la stima che tende all infinito e quindi dividiamo per i termini (*).
          E otteniamo per il teorema del doppio confronto che la sommatoria in mezzo vale 1.
        \end{flushleft}
        \begin{flushleft}
          Siano $f,g:\mathbb{N} \to \mathbb{R}$\\
          Def: f e g si dicono asintoticamente equivalenti scritte f$\cong$g, se:
          \begin{equation*}
            lim_{n\to +\infty} \frac{f(n)}{g(n)} =1
          \end{equation*}
          \begin{flushleft}
            L'esempio di prima e' asintoticamente equivalente
          \end{flushleft}
          \begin{flushleft}
            Def: L'n-esimo numero armonico e':
            \begin{equation*}
              H_n=\sum_{i=1}^n \frac{1}{i}
            \end{equation*}
          \end{flushleft}
        \end{flushleft}
        \begin{flushleft}
          \textbf{E.g} Stimare $H_n$
        \end{flushleft}
        \begin{flushleft}
          La funzione $\frac{1}{x}$ e' continua e monotona decrescente, quindi per 4.3.1:
        \end{flushleft}
        \begin{equation*}
          1+\int_1^n \frac{1}{x}dx\geq \sum_{i=1}^n \frac{1}{i} \geq \frac{1}{n}+\int^n_1 \frac{1}{x}dx \quad \forall n \in \mathbb{N}
        \end{equation*}
        Ma
        \begin{equation*}
          \int^n_1 \frac{1}{x}dx=[ln(x)]^n_1
        \end{equation*}
        \begin{flushleft}
          Sostituendo otteniamo 
        \end{flushleft}
        \begin{equation*}
          1+ln(n) \geq \sum_{i=1}^n \frac{1}{i} \geq \frac{1}{n} +ln(n) \quad \forall n \in \mathbb{N}
        \end{equation*}
        \begin{flushleft}
          Dividendo per il membro che tende a infinito piu' velocemente otteniamo
        \end{flushleft}
        \begin{equation*}
          \frac{1}{ln(n)} +1 \geq \frac{H_n}{ln(n)} \geq \frac{1}{n*ln(n)} +1
        \end{equation*}
        \begin{flushleft}
          Se facciamo il limite otteniamo che
        \end{flushleft}
        \begin{equation*}
          \frac{H_n}{ln(n)}=1 \to H_n=ln(n)
        \end{equation*}
        \subsection{Somme doppie}
        \begin{flushleft}
          Quanto e' grande:
        \end{flushleft}
        \begin{equation*}
          \sum^n_{i=1} H_i = \sum^n_{i=1} \sum_{j=1}^i \frac{1}{j}?
        \end{equation*}
        \begin{flushleft}
          Cosa stiamo sommando veramente? Stiamo sommando tutti i numeri della seguente tabella 
        \end{flushleft}
        \begin{tabular}{ |p{1cm}||p{1cm}||p{1cm}||p{1cm}||p{1cm}||p{1cm}||p{1cm}|  }
          \hline
            \multicolumn{7}{|c|}{Tabella} \\
          \hline
            $i/j$& 1& 2& 3& ...& n-1& n \\
          \hline
            1& 1/2& & & & &\\
            2& 1/1& 1/2& & & &\\
            3& 1/1& 1/2& 1/3 & & &\\
            $\vdots$& 1/1& & & & &  \\
            n-1& 1/1& 1/2& 1/3& ...& 1/n-1& \\
            n& 1/1& 1/2& 1/3& ...& 1/n-1& 1/n\\
          \hline
        \end{tabular}
        \begin{flushleft}
          Se sommiamo questi numeri colonna per colonna otteniamo
        \end{flushleft}
        \begin{equation*}
          n*(1+1+1+1...+1)+(n*-1)(\frac{1}{2}+\frac{1}{2}+...+\frac{1}{2})+...+(\frac{1}{n-1}+\frac{1}{n-1})+(\frac{1}{n})=
        \end{equation*}
        \begin{equation*}
          = \sum_{i=1}^n \frac{1}{i}(n-i+1)= \sum^n_{i=1} (\frac{n}{i}-\frac{i}{i}+\frac{1}{i})=
        \end{equation*}
        \begin{equation*}
          = n  \sum_{i=1}^n -\frac{1}{i}-\sum_{i=1}^n 1+ \sum_{i=1}^n \frac{1}{i} = nH_i -n +H_i=(n+1)H_i -n
        \end{equation*}
        \begin{flushleft}
          In particolare
        \end{flushleft}
        \begin{equation*}
          \sum^n_{i=1} H_i \cong n*H_n \quad \quad \text{se $n\to +\infty$}
        \end{equation*}
        \subsection{Prodotti}
        \begin{flushleft}
          Sia $f: \mathbb{N} \to \mathbb{R}_{>0}$. Quanto e' grande il prodotto
        \end{flushleft}
        \begin{equation*}
          \Pi^n_{i=1} f(i)?
        \end{equation*}
        \begin{flushleft}
          Abbiamo che 
        \end{flushleft}
        \begin{equation*}
          \Pi^n_{i=1} f(i)=exp(\sum^n_{i=1} ln(f(i)))
        \end{equation*}
        \begin{flushleft}
          E stimiamo la somma \\ 
          E.g: quanto e' grande $n!$? Abbiamo che
        \end{flushleft}
        \begin{equation*}
          n!=exp(\sum^n_{i=1} ln(i))
        \end{equation*}
        \begin{flushleft}
          Quindi stimiamo
        \end{flushleft}
        \begin{equation*}
          \sum^n_{i=1} ln(i)
        \end{equation*}
        \begin{flushleft}
          La funzione ln(x) e' continua e monotona crescente per $x \in \mathbb{P}$. Quindi possiamo applicare 4.3.1, abbiamo:
        \end{flushleft}
        \begin{equation*}
          \int^n_1 ln(x)dx \leq \sum^n_{i=1} ln(i) \leq ln(n)+\int^n_i ln(x)dx
        \end{equation*}
        \begin{flushleft}
          Sviluppando l integrale e sostituendo otteniamo
        \end{flushleft}
        \begin{equation*}
          nln(n)-n+1 \leq \sum^n_{i=1} ln(i) \leq ln(n) +ln(n)-n+1
        \end{equation*}
        \begin{equation*}
          e quindi
        \end{equation*}
        \begin{equation*}
          \frac{e^{nln(n)}}{e^{n-1}} \leq n! \leq \frac{e^{ln(n)+nln(n)}}{e^{n-1}} \quad \quad \text{cioe'}
        \end{equation*}
        \begin{equation*}
          \frac{n^n}{e^{n-1}}\leq n! \leq \frac{n^{n+1}}{e^{n-1}}
        \end{equation*}
        \subsubsection{Formula di Stirling}
        \begin{flushleft}
          Esiste una funzione $\epsilon: \mathbb{N} \to \mathbb{R}_{>0}$ tale che
        \end{flushleft}
        \begin{equation*}
          \frac{1}{12n+1} \leq \epsilon(n) \leq \frac{1}{12n} \quad \quad \forall n \in \mathbb{N}
        \end{equation*}
        \begin{flushleft}
          Questo valore approssimato equivale a
        \end{flushleft}
        \begin{equation*}
          n!=\sqrt{2\pi n}*(\frac{n}{2})^n*e^{\epsilon(n)}
        \end{equation*}
        \begin{flushleft}
          Dimostrazione Omessa $\square$
        \end{flushleft}
        \subsection{Notazioni Asintotiche}
        \begin{flushleft}
          Siano $f,g: \mathbb{N} \to \mathbb{R}$
        \end{flushleft}
        \begin{flushleft}
          Def: Si diche ce f e' un o-piccolo (little-oh) di g, scritto f=o(g), se:
        \end{flushleft}
        \begin{equation*}
          \lim_{n \to +\infty} \frac{f(n)}{g(n)}=0
        \end{equation*}
        \subsubsection{Lemma}
        \begin{flushleft}
          Siamo $a,b \in \mathbb{R}_{>0}$ e $a<b$ . Allora
        \end{flushleft}
        \begin{equation*}
          x^a=o(x^b)
        \end{equation*}
        \begin{flushleft}
          Dimostrazione: Chiara $\square$
        \end{flushleft}
        \subsubsection{Lemma}
        \begin{flushleft}
          Siano $a,b \in \mathbb{R}$ e $a>1$. Allora
        \end{flushleft}
        \begin{equation*}
          x^b=o(a^x)
        \end{equation*}
        \begin{flushleft}
          Dimostrazione: Vedi corso di Analisi $\square$
        \end{flushleft}
        \subsubsection{Lemma}
        \begin{flushleft}
          Sia $b\in \mathbb{R} \quad b>0$ Allora
        \end{flushleft}
        \begin{equation*}
          ln(x)=o(x^b)
        \end{equation*}
        \begin{flushleft}
          \textbf{Dimostrazione} sappiamo che $x<e^x \quad \forall x \in \mathbb{R} \Rightarrow ln(x)<x \quad \forall x \in \mathbb{R}_{>0}
          \Rightarrow ln(x^{\frac{b}{2}})<x^{\frac{b}{2}} \quad \text{per} \quad \forall x >0 \Rightarrow \frac{b}{2}ln(x)<x^{\frac{b}{2}}$ pertanto
        \end{flushleft}
        \begin{equation*}
          \frac{ln(x)}{x^b}<\frac{2}{b}*\frac{1}{x^{\frac{b}{2}}} \to 0 \quad \text{se $x\to +\infty \quad \square$}
        \end{equation*}
        \begin{flushleft}
          Def f si dice un o-grande (BIG-OH) di g,scritto f=O(g), se $\exists c \in mathbb{R}_{>0}$ e $N\in \mathbb{N}$ tali che
        \end{flushleft}
        \begin{equation*}
          \mid f(n) \mid \leq c*g(n) \quad \text{se $n \geq N$}
        \end{equation*}
        \begin{flushleft}
          E.g sia $f(x)=3x^4-2x^2+7$ Allora
        \end{flushleft}
        \begin{equation*}
          \lim_{x\to +\infty} \frac{3x^4-2x^2+7}{x^4}=3
        \end{equation*}
        \begin{flushleft}
          pertanto $\exists N\in \mathbb{N}$ tale che
        \end{flushleft}
        \begin{equation*}
          \frac{\mid 3x^4-2x^2+7 \mid}{x^4} \leq 3+\frac{1}{2}
        \end{equation*}
        \begin{flushleft}
          se $x>N$ ma allora
        \end{flushleft}
        \begin{equation*}
          \mid 3x^4-2x^2+7 \mid  \leq \frac{7}{2} x^4
        \end{equation*}
        \begin{flushleft}
          Allo stesso modo si dimostra che 
        \end{flushleft}
        \subsubsection{Teorema}
        \begin{flushleft}
          Siano $a_0,...,a_k \in \mathbb{R}$,$(k\in \mathbb{P} \quad a_k \neq 0)$ Allora
        \end{flushleft}
        \begin{equation*}
          a_k*x^k+...+a_1*x+a_0=O(x^k)
        \end{equation*}
        \subsubsection{Proprieta'}
        \begin{flushleft}
          Siano $f,g:\mathbb{N} \to \mathbb{R}_{>0}$ allora
        \end{flushleft}
        \begin{enumerate}
          \item $f=o(g) \Rightarrow f=O(g)$
          \item $f \cong g \Rightarrow f=O(g)$
        \end{enumerate}
        \textbf{Dimostrazioni:}
        \begin{enumerate}
          \item se f=o(g) $\Rightarrow \lim_{n \to +\infty} \frac{f(n)}{g(n)}=0 \Rightarrow \exists N \in \mathbb{N}$ tale che 
            \begin{equation*}
              \frac{f(n)}{g(n)}\leq \frac{1}{2}
            \end{equation*}
            Se $n\geq N \Rightarrow \mid f(n) \mid \leq \frac{1}{2}g(n) \quad \text{se $n>N \Rightarrow f=O(n)$}$
          \item Se $f \cong g \Rightarrow \lim_{n \to +\infty} \frac{f(n)}{g(n)}=1
            \Rightarrow \exists M \in \mathbb{N}$
            \begin{equation*}
              \frac{f(n)}{g(n)} \leq \frac{3}{2}
            \end{equation*}
            Se $ n \geq M \Rightarrow f=O(g) \square$
        \end{enumerate}
        \begin{flushleft}
          Oss:Siano $f,g: \mathbb{N} \to \mathbb{R}_{>0}$. Allora
        \end{flushleft}
        \begin{flushleft}
          f=o(g) non implica g=O(f) (per esempio f(n)=n e g(n)=$n^3$ $\forall n \in \mathbb{N} \Rightarrow f=o(g) ma g\neq O(f)$
        \end{flushleft}
        \begin{flushleft}
          Def: f si dice Omega di g, scritto $f=\Omega(g)$ se $\exists c \in \mathbb{R}_{>0} \quad \exists N \in \mathbb{N}$ tali che
        \end{flushleft}
        \begin{equation*}
          f(n) \geq c*g(n) \quad \text{se $n>N$}
        \end{equation*}
        \subsubsection{Proprieta'}
        \begin{equation*}
          f=O(g) \iff g=\Omega(f)
        \end{equation*}
        \begin{flushleft}
          Dimostrazione: abbiamo che 
        \end{flushleft}
        \begin{itemize}
          \item $f=O(g) \iff \exists c>0 \land N \in \mathbb{N}$ tali che $f(n) \leq c*g(n)$
          \item $\forall n \in \mathbb{N} \iff \exists c>0 \land N \in \mathbb{N}$ tali che $g=\Omega(f) \square$
        \end{itemize}
        \begin{flushleft}
          Oss: f=O(g) non implica g=0(f) (f(n)=n,g(n)=$n^2$)
        \end{flushleft}
        \begin{flushleft}
          Def si dice che f e' un teta di g scritto, $f=\Theta(g)$, se f=O(g) e g=o(f)
        \end{flushleft}
        \begin{flushleft}
          Oss: Sia $f: \mathbb{N} \to \mathbb{R}_{>0}$ tale che $f=\Theta(n^3) \Rightarrow \exists c_1,c_2>0 \land \exists N \in \mathbb{N}$ tale che
        \end{flushleft}
        \begin{equation*}
          c_1*n^3 \leq f(n) \leq c_2*n^3
        \end{equation*}
        \begin{flushleft}
          $\forall n\leq N$ quindi se n radoppia $\Rightarrow f(n)$ si moltiplica per 8.
        \end{flushleft}
        \subsection{Quanto e' grande infinito}
        \begin{flushleft}
          Siano A e B insiemi \\ 
          Def: si dice che A e B hanno la stessa cardinalita', scritto $\mid A \mid = \mid B \mid$ se $\exists f:A\to B$, f biezione. Ricordiamo che:
        \end{flushleft}
        \begin{equation*}
          P(A)=\{S: S\subseteq A\}
        \end{equation*}
        \subsubsection{Teorema di Kantor}
        \begin{flushleft}
          Sia A un insieme allora:
        \end{flushleft}
        \begin{equation*}
          \mid A \mid \neq \mid P(A) \mid
        \end{equation*}
        \begin{flushleft}
          Dimostrazione: PEr assurdo sia $f: A \to P(A)$, f biezione. Sia:
        \end{flushleft}
        \begin{equation*}
          B=\{ x\in A: x \notin f(x)\}
        \end{equation*}
        \begin{flushleft}
          Allora $B \subseteq A \Rightarrow \exists y \in A$ tale che f(y)=B f surriettiva:
        \end{flushleft}
        \begin{itemize}
          \item Se $y\in f(y) \Rightarrow y\in B \Rightarrow y \notin f(x) \Rightarrow$ assurdo
          \item Se $y\notin f(y) \Rightarrow y\in B \Rightarrow y \in f(x) \Rightarrow$ assurdo
        \end{itemize}
        \begin{flushleft}
          Def: Si dice che la cardinalita' di A e'minore o uguale alla cardinalita' di B se $\exists f A \to B$, f inniettiva, scritto $\mid A \mid \leq \mid B \mid$
        \end{flushleft}
        \begin{flushleft}
          Def $\mid A \mid \leq \mid P(A) \mid$ pertanto
        \end{flushleft}
        \begin{equation*}
          \mid \mathbb{N} \mid < \mid P(\mathbb{N}) \mid <\mid P(P(\mathbb{N}))\mid < ...
        \end{equation*}
        \begin{flushleft}
          Ci sono infiniti infiniti
        \end{flushleft} 
        \section{CAD e Grafi}
        \subsection{Grafoi}
        \begin{flushleft}
          Sia A insieme e sia $k\in \mathbb{N}$ poniamo
        \end{flushleft}
        \begin{equation*}
          \begin{pmatrix}
            A \\ 
            k
          \end{pmatrix}= \{ S \subseteq A: \mid S \mid  =k \}
        \end{equation*}
        \begin{flushleft}
          Def: un grafo e' una coppia G=(v,e), dove v e' un'insieme, detto insieme dei vertici e $'e' \subseteq \begin{pmatrix} V \\ 2\end{pmatrix}$
          detto insieme dei lati (o spigoli, o edges, o archi)
        \end{flushleft}
        \begin{flushleft}
          Si rappresenta graficamente un grafo identificando ogni vertice di 'e' con un punto del primo e collegando 2 punti con un segmento, anche
          curvilineo, se i 2 vertici corrispondenti sono un lato \\
          E.g
        \end{flushleft}
        \begin{equation*}
          G=([5],\{\{1,2\},\{1,3\},\{4,5\},\{5,3\},\{2,4\}\})
        \end{equation*}
        Dove
        \begin{itemize}
          \item '[5]' rappresenta i numero di vertici, cioe' punti che poi andranno collegati
          \item \{\{1,2\},\{1,3\},\{4,5\},\{5,3\},\{2,4\}\} questo insieme rappresenta tutti collegamenti che si vogliono fare
        \end{itemize}
        \subsubsection*{Definizioni}
        \begin{itemize}
          \item Def: il grafo vuoto su n vertici e'
            \begin{equation*}
              N_n=([n],\o)
            \end{equation*}
          \item Def: il grafo completo su n vertici e'
            \begin{equation*}
              K_n=([n],\begin{pmatrix}
                [n] \\ 2
              \end{pmatrix})
            \end{equation*} dove il coefficiente binomiale rappresenta tutte le possibili combinazioni di come si possono collegare i punti sul grafo
          \item Def: un cammino in G e' una sequenza di vertici $\to (x_0,x_1,x_2,...,x_k)\in V^{k+1}$ tale che $\{x_i,x_{i+1}\}\in E$ per $(\forall i=1,...,k-1)$,
            K si dice la lunghezza del cammino
          \item Def:un sentiero in G e' un cammino $(x_0,...,x_k)$ tale che $0\leq i, j\leq k,i \notin k \Rightarrow x_i \neq x_j$
          \item Def:un cammino chiuso in G e'un cammino $(x_0,...,x_k)$ tale che $x_0=x_k$
          \item Def:un ciclo in E e' un cammino chiuso $(x_0,...,x_k)$ tale che $(x_1,...,x_k)$ e' un sentiero
          \item Def: G e' connesso se per $\forall x,y \in V, x\neq y \Rightarrow \exists$ un cammino $(x_0,...,x_k)$ tale che $x_0=x \land x=y$
        \end{itemize}
        \begin{flushleft}
          Sia G=(V,E) un grafo
        \end{flushleft}
        \subsection*{Definizioni}
        \begin{itemize}
          \item G e' ciclico se ha cicli di lunghezza >2
          \item G e' un albero se G e' connesso e ciclico
          \item G e' una foresta se G e' ciclico (insieme di alberi)
        \end{itemize}
        \begin{flushleft}
          SIa $S \subseteq V$
        \end{flushleft}
        \subsection*{Definizioni}
        \begin{itemize}
          \item S e' indipendente se $x,y\in S,x\neq y \Rightarrow \{x,y\} \notin E$
          \item S e' completo (o una clique) se $x,y \in S,x\neq y \Rightarrow \{x,y\} \in E$
          \item Sia $x \in V$ Def: il grado di x e'
            \begin{equation*}
              d(x)=\mid \{ y\in V:\{x,y\}\in E\} \mid
            \end{equation*}
        \end{itemize}
        \begin{flushleft}
          Siano G=(V,E) e H=(W,F) due grafi
        \end{flushleft}
        \begin{flushleft}
          Def: Si dice che G e H sono isomorfi se $\exists f: V\to W$,f biezione, tale che, per $\forall x,y \in Vw$
        \end{flushleft}
        \begin{equation*}
          \{ x,y \} \in E \iff \{ f(x),f(y) \} \in F
        \end{equation*}
        \begin{flushleft}
          Intituitivamente Posso cambiare i nomi dei vertici di H in modo che G=H
        \end{flushleft}
        \begin{flushleft}
          Notazioni: se G e H sono isomorfi si scrive $G \cong H$
        \end{flushleft}
        \begin{flushleft}
          Sia P una proprieta' che un grafo puo' avere oppure no
        \end{flushleft}
        \begin{flushleft}
          Def: P si dice invariante per isomorfosi se presi comunque due grafi G e H tali che
        \end{flushleft}
        \begin{equation*}
          (G \quad \text{ha} \quad P) \iff (H \quad \text{ha} \quad P) \Rightarrow G \cong H
        \end{equation*}
        \begin{flushleft}
          E.g
        \end{flushleft}
        \begin{itemize}
          \item Avere 18 vertici e' invariante isomorfia
          \item Avere 14 lati e' invariante per isomorfia
          \item Avere un ciclo di lunghezza 4: SI
          \item Avere un vertice di grado 5: SI
          \item Avere 4 come vertice: non e' invariante per isomorfia
        \end{itemize}
        \subsubsection{Lemma}
        \begin{flushleft}
          Sia G=(V,E) un grafo finito. Allora:
        \end{flushleft}
        \begin{equation*}
          \sum_{x \in V} d(x)=2*\mid E \mid
        \end{equation*}
        \begin{flushleft}
          Dimostrazione: Abbiamo che
        \end{flushleft}
        \begin{equation*}
          \mid \{(x,e) \in VxE: x\in e\} \mid = \sum_{x \in V} \mid \{ e\in E: x\in e\} \mid = \sum_{x\in V} d(x)
        \end{equation*}
        D'altronde
        \begin{equation*}
          \mid \{(x,e) \in VxE: x\in e\} \mid = \sum_{e \in E} \mid \{x \in V: x\in e \} \mid = \sum_{e \in E} 2 \square
        \end{equation*}
        \begin{flushleft}
          Sia G=(V,E) un grado
        \end{flushleft}
        \begin{flushleft}
          Def: G e' bipartito se $\exists V_1,V_2 \subseteq V$ tali che $V_1$ e $V_2$ sono indipendenti e
        \end{flushleft}
        \begin{equation*}
          V=V_1 \Cup V_2
        \end{equation*}
        \begin{flushleft}
          E.g: un esagono e' bipartito perche' mettendo nell'insieme $V_1$ 3 vertici che non si collegano con dei lati e nell insieme $V_2$ la stessa cosa
          se questa cosa e possibile allora e' bipartito
        \end{flushleft}
        \subsection{Accoppiamento}
        \begin{flushleft}
          Sia G=(V,E) un grafo 
        \end{flushleft}
        \begin{flushleft}
          Def: Un accoppiamento di G (matching) e' un sottoinsieme $M \subseteq E$ tale che
        \end{flushleft}
        \begin{equation*}
          l_1,l_2 \in M \Rightarrow l_1 \cap l_2 = \o
        \end{equation*}
        \begin{flushleft}
          E.g: un grafo a forma di esagono e' un accoppiamento invece un pentagono no
        \end{flushleft}
        \begin{flushleft}
          Sia G=(V,E) un grafo bipartito (siano $V_1,V_2$ come nella definizione di grafo bipartito
        \end{flushleft}
        \begin{flushleft}
          Def: un accoppiamento di $V_1$ in $V_2$ e' un accoppiamento M di G tale che $\forall u\in V_1 \Rightarrow \exists v \in V_2$ tale che $\{u,v\} \in M$
        \end{flushleft}
        \begin{flushleft}
          (mettere esempio (?))
        \end{flushleft}
        \begin{flushleft}
          Sia G=(V,E) un grafo bipartito come quello sopra. Poniamo
        \end{flushleft}
        \begin{equation*}
          N_G(S)=\{v \in V_2: \exists u\in S \quad \text{tali che }\{u,v\}\in E \}
        \end{equation*}
        \begin{flushleft}
          (Mettere esempio (?))
        \end{flushleft}
        \subsubsection{Teorema di Hall (Teorema di matrimonio)}
        \begin{flushleft}
          Sia G=(V,E) un grafo bipartito come sopra. Allora esiste un accoppiamento di $V_1$ in $V_2$ se e solo se
        \end{flushleft}
        \begin{equation*}
          \mid S \mid \leq \mid N_G(S) \mid
        \end{equation*}
        \begin{flushleft}
          $\forall S \subseteq V_1$
        \end{flushleft}
        \begin{flushleft}
          Dimostrazione per induzione su $\mid E \mid$
        \end{flushleft}
        \begin{enumerate}
          \item $\mid S \mid < \mid N_G(S) \mid$ per $\forall S \not \subseteq V_1$
            \begin{flushleft}
              Sia $x\in V_1$, sia $y\in N_G(\{x\})(\Rightarrow \{x,y\}\in E)$ Sia
            \end{flushleft}
            \begin{equation*}
              H=(V/ \{x,y\}, E/E*)
            \end{equation*}
            \begin{flushleft}
              Dove $E*=\{l in E: x\in l \lor y\in l \}(\Rightarrow $H e' bipartito)
            \end{flushleft}
            \begin{flushleft}
              Sia $T \subseteq V_1/\{x\}$
            \end{flushleft}
            \begin{flushleft}
              Allora $N_H(T)=$
            \end{flushleft}
            \begin{itemize}
              \item $N_G(T), \text{ se } y\notin N_G(T)$
              \item $N_G(T)/\{y\}, \text{ se } y\in N_G(T)$
            \end{itemize}
            \begin{flushleft}
              Quindi
            \end{flushleft}
            \begin{equation*}
              \mid T \mid \leq \mid N_G(T) \mid -1 \leq \mid N_H(T) \mid
            \end{equation*}
            \begin{flushleft}
              Pertanto per induzione, esiste un accoppiamento di $V_1 / \{x\}$ in $V_2/ \{y\}$ in H $\Rightarrow$ un accoppiamento di $V_1$ in $V_2$ in G.
            \end{flushleft}
          \item $\exists S \not \subseteq V_1$ tale che
            \begin{equation*}
              \mid S \mid = \mid N_G(S) \mid
            \end{equation*}
            \begin{flushleft}
              Siano
            \end{flushleft}
            \begin{equation*}
              R=(S \cup N_G(S), E_R) \quad \text{dove} \quad E_R=\{l\in E: l \cap S \neq \o \land l\cap N_G(S) \neq \o \}
            \end{equation*}
            \begin{equation*}
              B=((V_1/S)\cup (V_2/N_G(S)), E_B) \quad \text{dove} \quad E_B=\{l\in E: l \cap S = \o \land l\cap N_G(S) = \o \}
            \end{equation*}
            \begin{flushleft}
              Notiamo che
            \end{flushleft}
            \begin{equation*}
              \{ l \in E: l\cap S=\o \land l\cap N_G(S)=\o\} = \o
            \end{equation*}
            \begin{flushleft}
             Sia $T\subseteq S$ allora  
            \end{flushleft}
            \begin{equation*}
              \mid N_R(T) \mid = \mid N_G(T) \mid \geq \mid T \mid
            \end{equation*}
            \begin{flushleft}
              $\geq$ = ipotesi
            \end{flushleft}
            \begin{flushleft}
              Pertanto per induzione $\Rightarrow \exists$ un accoppiamento di S in $N_G(S)$ in R
            \end{flushleft}
            \begin{flushleft}
              Sia $T \subseteq V_1/S$ Allora 
            \end{flushleft}
            \begin{equation*}
              \mid T \mid + \mid S \mid = \mid T \cup S \mid \leq \mid N_G(T \cup S) \mid =
            \end{equation*}
            \begin{equation*}
              = \mid N_G(S) \cup N_B(T) \mid = \mid N_G(S) \mid + \mid N_B(T) \mid = \mid S \mid + \mid N_B(T) \mid \Rightarrow \mid T \mid \leq \mid N_B(T) \mid 
            \end{equation*}
            \begin{flushleft}
              Se $x\in N_G(S) \Rightarrow$ ok
            \end{flushleft}
            \begin{equation*}
              x\notin N_G(S) \Rightarrow x\in N_G(S\cup T) / N_G(S) \Rightarrow x\in V_2 / N_G(S) \land \exists y\in S \cup T
            \end{equation*}
            tale che
            \begin{equation*}
              \{x,y\}\in E \Rightarrow y \in V/S \Rightarrow \{x,y\}\in E_B \Rightarrow x\in N_B(T)
            \end{equation*}
            \begin{flushleft}
              Pertanto per induzione $\Rightarrow \exists$ accopiamnto di $V_1/S$ in $V_2/N_G(S)$ in B
            \end{flushleft}
            \begin{flushleft}
              Quindi $\Rightarrow \exists$ accoppiamento di $V_1$ in $V_2$ in G. Il viceversa e' facile $\square$
            \end{flushleft}
        \end{enumerate}
        \begin{flushleft}
          Sia G=(V,E) un grafo bipartito come sopra
        \end{flushleft}
        \begin{flushleft}
          Def: G si dice ristretto (o legato) nei gradi da $V_1$ a $V_2$ se 
        \end{flushleft}
        \begin{equation*}
          d(x) \geq d(y)
        \end{equation*}
        \begin{flushleft}
          per $\forall x \in V_1$ e $\forall y \in V_2$
        \end{flushleft}
        \subsubsection{Proprieta'}
        \begin{flushleft}
          Sia G=(V,E) un grafo bipartito come sopra e legato nei gradi da $V_1$ a $V_2$. Allora $\Rightarrow \exists$ un accoppiamento di $V_1$ in $V_2$
        \end{flushleft}
        \begin{flushleft}
          Dim: Sia $d\in \mathbb{P}$ tale che
        \end{flushleft}
        \begin{equation*}
          d(x) \geq d \geq d(y) \quad \text{per} \quad \forall x \in V_1 \land \forall y \in V_2
        \end{equation*}
        Sia $S \subseteq V_1$ allora
        \begin{equation*}
          d*\mid S\mid \leq \sum_{d\in S} d(x)=\mid E_S \mid \quad \text{dove} \quad E_S=\{l\in E:l\cap S\neq \o\}
        \end{equation*}
        Ma $E_S \subseteq E_{N_G(S)}$ dove
        \begin{equation*}
          E_{N_G(S)}=\{l \in E: l\cap N_G(S)\neq \o\}
        \end{equation*}
        \begin{equation*}
          \mid E_{N_G(S)} \mid = \sum_{y\in N_G(S)} d(y) \leq d* \mid N_G(S) \mid
        \end{equation*}
        Concludendo
        \begin{equation*}
          d*\mid S \mid \leq d*\mid N_G(S) \mid \Rightarrow \exists \quad \text{accoppiamento di $V_1$ in $V_2$ $\square$}
        \end{equation*}
        \begin{flushleft}
          Questo vero per 5.2.1
        \end{flushleft}
        \begin{flushleft}
          Sia G=(V,E) un grafo, e sia $d\in \mathbb{P}$
        \end{flushleft}
        \begin{flushleft}
          Def: G si dice regolare (di grado i) se d(x)=d per $\forall x\in V$
        \end{flushleft}
        \subsubsection{Proprieta'}
        \begin{flushleft}
          Sia G=(V,E) un grafo bipartito regolare, Allora $\Rightarrow \exists$ accoppiamento di $V_1$ in $V_2$ e $\exists$ accoppiamento di $V_2$ in $V_1$
        \end{flushleft}
        \begin{flushleft}
          Dimostrazione: segue subito da 5.2.2 $\square$
        \end{flushleft}
        \subsection{Colorazioni}
        \begin{flushleft}
          Sia G=(V,E) un grafo, e sia $f:V\to [n] \quad (n \in \mathbb{P}$
        \end{flushleft}
        \begin{flushleft}
          Def: f si dice una colorazione di G (in al piu n colori) Se:
        \end{flushleft}
        \begin{equation*}
          x \neq y,\{x,y\}\in E\Rightarrow f(x)\neq f(y), \forall x,y \in V
        \end{equation*}
        \begin{flushleft}
          In tal caso si dice che G e' colorabile (con al piu n colori)
        \end{flushleft}
        \begin{flushleft}
          Def: il numero cromatico di G, scritto $\chi$(G), e' il piu piccolo $n\in \mathbb{P}$ per cui G puo' essere colorato con al piu' n colori
        \end{flushleft}
        E.g
        \begin{flushleft}
          Un esagono si puo' colorare con 2 colori $\Rightarrow \quad \chi(G)=2$
        \end{flushleft}
        \begin{flushleft}
          Un pentagono si puo' colorare con 3 colori $\Rightarrow \quad \chi(G)=3$
        \end{flushleft}
        \begin{flushleft}
          Oss: $\chi(K_n)=n,\chi(N_N)=1 \quad \forall n \in \mathbb{P}$
        \end{flushleft}
        \subsubsection{Proprieta'}
        \begin{flushleft}
          SIa G=(V,E) un grafo allora
        \end{flushleft}`
        \begin{equation*}
          \chi(G) \leq max_{v \in V}\{d(v)\}+1
        \end{equation*}
        \begin{flushleft}
          Questa e' una stima in quanti colori si possono colorare certi grafi
        \end{flushleft}
        \begin{flushleft}
          Dimostrazione:Induzione su $\mid V \mid \geq 1$.
        \end{flushleft}
        \begin{flushleft}
          Se $\mid V \mid =1 \Rightarrow \chi(G)=1 \leq 0+1=max\{d(v)\}+1$ OK
        \end{flushleft}
        \begin{flushleft}
          Sia il risultato vero per tutti i grafi con $\leq n-1$ vertici
        \end{flushleft}
        \begin{flushleft}
          Sia $\mid V \mid =n$.Sia $x\in V$.Sia $G'=(V/\{x\},E')$ dove $E'=\{l\in E: x \notin l\}$
        \end{flushleft}
        \begin{flushleft}
          Allora per induzione
        \end{flushleft}
        \begin{equation*}
          \chi(G)\leq max_{v\in V/\{x\}} \{d_G(v)\}+1\leq max_{v\in V/\{x\}} \{d_G(v)\}+1\leq max_\{v\in V\}\{d_G(v)\}+1
        \end{equation*}
        \begin{flushleft}
          Sia $k=max_\{v\in V\}\{d_G(v)\}+1 \Rightarrow$ G' e' colorabile con al piu' k colori.
        \end{flushleft}
        \begin{flushleft}
          $\{v_1,...,v_2\}=\{u \in V:\{u,x\}\in E\} \Rightarrow r=k-1 \Rightarrow r<k$
        \end{flushleft}
        \begin{flushleft}
          Ma $v_1,...,v_r$ usano al piu' piccolo r colori $\Rightarrow$ posso assegnare ad x uno dei k-1 colori rimanenti. Quindi G e' colorabile in k colori $\Rightarrow \chi(G)\leq k \square$
        \end{flushleft}
        \subsection{grafi diretti}
        \begin{flushleft}
          Def:un grafo diretto (o disgrafo) e' una coppia D=(V,A) dove V e' un insieme (detto insieme dei vertici, o spigoli orientati, o lati diretti,o freccie)
          Se $(u,v)\in A$ scriviamo $u \to v$.
        \end{flushleft}
        E.g:
        \begin{flushleft}
          Facebook e' un grafo, X (ex twitter) e' un grafo diretto
        \end{flushleft}

      \end{document}
