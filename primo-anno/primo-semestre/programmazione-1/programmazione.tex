\documentclass{article}
\title{Programmazione}
\usepackage{listings}
\begin{document}
    \section{Python}
    \subsection{Esercizio: controllare se la parola e palindroma}
    \begin{flushleft}
       Una parola palindroma essenzialmente e' una parola che si puo leggere in entrambi i vers E.g: otto,anna,radar,etc
       \subsection{L'algoritmo}
       \begin{flushleft}
        Per scrivere questo programma basterebbe far partire un loop dalla lunghezza della stringa, e salvare ogni carattere in quell'ordine
        in un altra stringa e infine confrontarle. Ma questo implica un sacco di passaggi quindi un altro modo per farlo che riduce il tempo
        di questo e' creare una funzione che \textbf{simmetricamente} controlla la prima lettera e l'ultima lettera se sono uguali.Non appena
        la funzione legge 2 lettere diverse ritorna il valore falso alla funzione, invece se non trova disuguaglianze continua finche non controlla
        tutta la stringa e a quel punto restuira alla funzione il valore booleano vero.\\
        \begin{lstlisting}[language=Python]
            def palindromo(a):
                i,n = 0, len(a)
                while i < n // 2:
                    if a[i] != a[-i-1]: # indici negativi
                        return False
                    i+=1
                return True
        \end{lstlisting}
        a[-i-1] rappresenta l'uso dell indici negativi in python, infatti per rappresentare l ultimo elemento di una stringa/array
        si possono usare gli indici negativi.
        \begin{flushleft}
            E.g: [-1] $\to$ ultima posizione della stringa, [-2] $\to$ penultima posizione della stringa
        \end{flushleft}
        \subsection{Come estrarre una porzione di stringa/array}
        \begin{flushleft}
            Si puo estrapolare un certo contenuto di una stringa immettendo la posizione di inizio e la posizione di fine
            \begin{lstlisting}[language=Python]
                x="0123456789"
                print(x[3:7])
            \end{lstlisting}
            Il 3 indicato sopra rappresenta l inzizio e il rappresenta la fine e questo codice come risultato riporta: 3456\\
            Si possono anche impostare i passi, cioe' quanti salti deve fare ogni volta che legge un carattere (defualt=1)
            \begin{lstlisting}[language=Python]
                x="0123456789"
                print(x[3:7:2]
            \end{lstlisting}
            Questo stampera come risultato solo "35": perche quando arriva a 3 salta il 4 e scrive il 5.
        \end{flushleft}
       \end{flushleft}
    \end{flushleft}
\end{document}
