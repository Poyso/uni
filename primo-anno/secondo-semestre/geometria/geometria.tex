\documentclass{article}
\usepackage{amsmath}
\usepackage{amssymb}
\usepackage{amsfonts}
\usepackage{inputenc}
\title{Geometria e Algebra}
\begin{document}
\section{Geometria affine (richiami)}
\begin{flushleft}
	In geometria useremo dei spazi in cui lavoreremo che sono: retta, piano e spazio
\end{flushleft}
\begin{itemize}
	\item retta (euclidea) affine $(A^1)$
	\item piano affine ($\pi$ = piano) ($A^2$) $\pi \supseteq r$
	      \begin{flushleft}
		      $r_1$ e $r_2$ rette in $\pi$ sono:
	      \end{flushleft}
	      \begin{itemize}
		      \item inicidenti se $r_1 \cap r_2=\{ P \}$ (p punto)
		      \item parallele se $r_1 \cap r_2=\emptyset$ e se $ r_1 =r_2$
	      \end{itemize}
	\item spazio affine
	      \begin{flushleft}
		      I suoi sottoinsiemi notevoli sono:
	      \end{flushleft}
	      \begin{itemize}
		      \item punti (P)
		      \item rette (r)
		      \item piani ($\pi$)
	      \end{itemize}
	      \begin{flushleft}
		      Ora le varie regole
	      \end{flushleft}
	      \begin{itemize}
		      \item $\pi$ e $\pi$' nello spazio sono:
		            \begin{itemize}
			            \item incidenti se $\pi \cap \pi ' =r$
			            \item paralleli se $\pi \cap \pi ' =\emptyset$ e se $\pi = \pi '$
		            \end{itemize}
		      \item r e $\pi$ nello spazio sono:
		            \begin{itemize}
			            \item incidenti se $\pi \cap r = \{ P \}$
			            \item paralleli se $\pi \cap r = \emptyset$ e se $r \subseteq \pi$
		            \end{itemize}
		      \item r e r' sono:
		            \begin{itemize}
			            \item complanari se $\exists \pi: \quad \pi \subseteq r \land \pi \subseteq r'$
			            \item sgherbe se $r \cap r' = \emptyset$ e se $r \nparallel r'$
		            \end{itemize}
	      \end{itemize}
\end{itemize}

\section{Vettori orientati}
\begin{flushleft}
	I vettori orientati sono segmenti che partono da un punto di origine O e arrivano ad un punto P, se si vogliono avere piu' vettori
	questi devono partire tutti dallos stesso punto O
\end{flushleft}
\begin{flushleft}
	Def: Esiste una funzione $\Phi_o : A_o \rightarrow V_o^{1/2/3}$
\end{flushleft}
\begin{equation*}
	\Phi_o(P)=\overrightarrow{OP}
\end{equation*}
\begin{flushleft}
	Nota: La funzione $\Phi_o$ e' biettiva
\end{flushleft}
\subsection{Somma tra vettori}
\begin{flushleft}
	Def:$\forall v = \overrightarrow{OP}$ e $v' = \overrightarrow{OP}' in V_o^{1/2/3} \to \exists ! v''=\overrightarrow{OP}''$
\end{flushleft}
\begin{flushleft}
	Per la somma tra vettori si costruisce un parallelogramma con le parallele di entrambi i vettori di cui si intende sommare
	e poi si traccia una diagonale dal punto O fino al vertice che si formera e questa diagonale rappresentera' la somma di 2 vetttori
\end{flushleft}
\subsection{Proprieta' della somma in $V_o^{1/2/3}$}
\begin{enumerate}
	\item La somma e' associativa
	\item $\exists \overrightarrow{OO} \in V_o^{1/2/3}$ e' un elemento neutro della somma
	\item $\forall \overrightarrow{OP}, \exists \overrightarrow{OP}' \in V_o^{1/2/3}$ tale che
	      \begin{equation*}
		      \overrightarrow{OP} +\overrightarrow{OP}' = \overrightarrow{OO}=\overrightarrow{OP}' +\overrightarrow{OP}
	      \end{equation*}
	\item La somma e' commutativa
\end{enumerate}
\section{Moltiplicazione dello scalare}
\begin{flushleft}
	Def: $\forall v = \overrightarrow{OP} \in V_o^n$ (n=1,2,3)
	$\forall t \in \mathbb{R}, \exists !t: v=t*\overrightarrow{OP} \in V^n_o$ tale che
	$t*\overrightarrow{OP}=\overrightarrow{OP_t}$ dove $P_t \in A^t$ e' dato da
\end{flushleft}
\begin{itemize}
	\item Se $t=0$ $\to$ $P_t:= 0 \to t*\overrightarrow{OP}=\overrightarrow{OO}$
	\item Se $t>0$ $\to$ $P_t$ e' nella retta di $\overrightarrow{OP}$:
	      \begin{equation*}
		      \frac{lungh(\overrightarrow{OP_t})}{lungh(\overrightarrow{OP})} = t = |t|
	      \end{equation*}
	\item Se $t<0$:
	      \begin{equation*}
		      \frac{lungh(\overrightarrow{OP_t})}{lungh(\overrightarrow{OP})} = -t = |t|
	      \end{equation*}
\end{itemize}
\begin{flushleft}
	Tutto questo e' vero se $\overrightarrow{OP} \neq \overrightarrow{OO}$
	se invece $\overrightarrow{OP} = \overrightarrow{OO}$ allora $t*\overrightarrow{OO}=\overrightarrow{OO}$
\end{flushleft}
\begin{flushleft}
	Quindi esiste una funzione:
\end{flushleft}
\begin{itemize}
	\item $\mathbb{R}xV_o^n \to V_o^n$
	\item $(t,v) \to t*v$
\end{itemize}
\subsection{Proprieta' della moltiplicazione}
\begin{flushleft}
	$\forall v',v'' \in V_o^n$\\
	$\forall t',t'' \in \mathbb{R}$
\end{flushleft}
\begin{enumerate}
	\item $t' * (t''*v)=(t''*t')*v$
	\item $1*v=v$
	\item $t' + (t''+v)=(t''+t')+v$
	\item $t*(v'+v'')=(t*v')+(t*v'')$
\end{enumerate}
\end{document}

